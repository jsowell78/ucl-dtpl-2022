
\documentclass[11pt]{letter}

\usepackage[letterpaper, portrait, margin=1in]{geometry}

\usepackage[colorlinks = true,
            linkcolor = blue,
            urlcolor  = blue,
            citecolor = blue,
            anchorcolor = blue]{hyperref}

\usepackage{paralist}

\usepackage{enumitem}
\setlist[itemize]{noitemsep}
\setlist[enumerate]{noitemsep}

\longindentation=2in

\usepackage{etoolbox}

\setlength{\parskip}{0.3\baselineskip}

\makeatletter
% If pagestyle is firstpage, remove old placement of date
\patchcmd\opening{{\raggedleft\@date\par}}{}{}{}
% If pagestyle is empty, remove vertical space after addess
\patchcmd\opening{\fromaddress \\*[1\parskip]}{\fromaddress}{}{}
% If pagestyle is empty, remove old placement of date
\patchcmd\opening{\@date \end{tabular}\par}{\end{tabular}\par}{}{}


\name{Jesse H. Sowell II}
\signature{Jesse H. Sowell II}

\def\fromaddress{\null}

\address
{
  2201 Crescent Pointe Parkway, Apt. 4302 \\
  College Station, TX  77845 \\
  m: +1 517 214 1900 \\
  e: \href{mailto:jesse.sowell@gmail.com}{jesse.sowell@gmail.com}
}



\begin{document}

\begin{letter}
{
  Department of Security and Crime Science \\ 
  University College London
}

\opening{Dear Faculty Search Committee,}

I am writing to express my interest in the position of Lecturer in the Department of Security and Crime Science at University College London.  
%
I am currently an Assistant Professor of International Affairs in the Bush School of Government and Public Service at Texas A\&M University.  
%
I hold a PhD in Technology, Management, and Policy from MIT's Engineering Systems Division.
% %
% Previously, 
% % I completed my PhD in Technology, Management at MIT's Engineering Systems
% % Division in 2015 and 
% I was a Postdoctoral Fellow at Stanford's Center for International Security and Cooperation (CISAC).  
%
My research focuses on the political economy of Internet policy and security, in particular the norms and best practices of private transnational cybersecurity regimes and how these actors engage with state-based security organizations and law enforcement.
%
My interdisciplinary background integrates political economy, computer science, and operations strategy. 
%
My empirical work builds on extensive engagement with global expert practitioner communities that ensure the security and stability of the Internet's infrastructure and online platforms.
%
% \textbf{Revise to focus on political economy}
%
% My political economy and international policy research the evolving relationship between technical epistemic communities and policy makers, drawing on co-regulatory approaches to integrating these actors' deep knowledge, capabilities, and capacities into domestic and international policy development, design, and implementation. 
%
I thrive in and embrace multidisciplinary environments.
%
I believe my intrinsically interdisciplinary background would be a rare and valuable complement to the research, teaching, and impact of the Department of Security and Crime Science and the Dawes Centre for Future Crime. 
%
% I thrive in and embrace multidisciplinary environments, and believe my background, research, and teaching is exceptionally aligned with this posting and the development objectives of the TechGov Centre.


My applied research is driven by real-world problems in today's digital platforms---crimeware-as-a-service, phishing and other transnational fraud campaigns, industrial and political espionage, and disinformation campaigns.
%
Transnational cybersecurity challenges cannot be resolved by policy formulation and implementation by a single stakeholder group or jurisdiction.
%
Remediating these kinds of sophisticated cybersecurity and cybercrime incidents requires combining technical, operational, organizational, and legal capabilities distributed amongst an equally diverse set of state and non-state, private actors.
%
Across specific cybersecurity challenges, my work examines how communities of expert practitioners manage transnational cybersecurity incidents as they emerge in real-time, the coproduction of knowledge and threat intelligence that supports these responses, and the barriers to collaboration between these on-the-ground cybersecurity experts and law enforcement.
%
Understanding how to systematically bridge these gaps contributes not only to the security and stability of today's Internet, but is also critical for keeping pace with innovative cybercriminals that have made a business of exploiting new and evolving technologies. 

%
% These kinds of systemic risks are characteristic of complex digital infrastructures (and more broadly, large scale complex engineering systems).
%
% Global communities of Internet operations and cybersecurity experts have demonstrated the ability to quickly identify and mitigate transnational cybersecurity incidents, but these private actors alone lack the domestic and international authority necessary to remediate through criminal investigations, legally enforced takedown orders, and prosecution.
% %
% Collaborations do occur, but remain ad hoc and informal.  
%
% A key challenge in my security work is understanding how to bridge these gaps by more systematically integrating these private actors' necessary capabilities and capacities into state-based domestic and international processes and governance systems.
%  contributing to not only the security and stability of today's Internet, but also bringing these private actors' unique capabilities and capacities to bear on new and continuously evolving cybersecurity threats such as crimeware-as-a-service, transnational fraud campaigns, industrial and political espionage, and disinformation campaigns.

My work addresses two key parts of this challenge. 
%
First, my empirical work builds on established theories of institutional design and political economy to understand how 
% the deep knowledge generated by consensus-based rule-making and governance mechanisms among 
transnational epistemic communities develop, and adapt, the norms, standards, and best practices they use to cope with uncertainties endemic in complex engineering systems such as the Internet.
%
Second, I evaluate the feedback loops necessary to systematically integrate this knowledge into the policy-making processes of international organizations and governance structures, while being cognizant of the politics of science and technology policy advice.
%
My fieldwork on cybersecurity and cybercrime focuses on the reactive, ad hoc relationships between private cybersecurity experts and law enforcement (both domestic and international), investigating how cybersecurity actors collaborate across firm and jurisdictional boundaries to identify emerging threats as they evolve, and the challenges facing collaboration with law enforcement.
%and the knowledge, organizational, and legal barriers to making these critical relationships more durable and systematic.  
%
I believe my work identifying and closing these knowledge gaps, along with my work on consensus-based decision making and credible knowledge assessment among technical and operational actors, is exceptionally aligned with the Centre for Future of Crime.


% Recently, I have applied this work to develop partnerships among transnational cybersecurity stakeholder groups and digital platform managers, helping these actors engage with the policy community on issues such as the implementation of the GDPR and institutional development enhancing connectivity and cybersecurity in regions of South America and Africa.


\textbf{Research} \vspace{0.2 \baselineskip} \newline %
%
% My research strategy has always been interdisciplinary. 
%
I started my academic life in computer science as a software engineer focusing on network security, but soon realized technical knowledge alone was insufficient to understand the complex sociotechnical dynamics shaping cybersecurity challenges.
%
My formal entry into the social sciences was through criminology and criminal justice, focusing on cybercrime, with an interest in informal norms and pro-social rule-breaking in the development of cybersecurity policy and best practices.
%%
My doctoral research at MIT combined international political economy, operations strategy, and computer science to conduct extensive fieldwork examining on-the-ground practices in Internet infrastructure management and cybersecurity.
%
I funded the last year of my dissertation work as the primary author on a Google Faculty Research Award (\$85,000). 
%
My current projects include 
%
\begin{enumerate}
  \item the Internet Infrastructure Mapping Project, examining the role of network interconnection platforms and submarine cables in the development of knowledge economies through case studies, mapping, and network analysis; 
  \item the Combined Capabilities Project (elaborated below, currently funded at \$30,000); 
  \item work on Internet consolidation (\href{https://www.tandfonline.com/doi/full/10.1080/23738871.2020.1754443}{published} in the \emph{Journal of Cyber Policy} and presented at \href{https://www.chathamhouse.org/2019/12/internet-consolidation-what-lies-beneath-application-layer}{Chatham House} in December 2019); and 
  \item IoT security and reputation, based on applications of my planned adaptation work (\href{https://link.springer.com/chapter/10.1007/978-3-030-05252-2_13}{book chapter}) in a \href{https://onlinelibrary.wiley.com/doi/full/10.1111/rego.12343}{recent article} on adaptive regulatory governance design for IoT security standards with Dr. Irina Brass (UCL STEaPP) in \emph{Regulation \& Governance}.
\end{enumerate}%

% My work characterizes the institutional capabilities and capacities of the understudied communities managing common platforms critical to Internet stability and security.
%
Broadly, my research explains and
evaluates the norms, values, and practices (i.e. institutions) that
sustain the core functions of the Internet's infrastructure.
%
My work examines the consensus-based coproduction of these institutions' expert knowledge, how it sustains the capabilities and capacities necessary to adapt to the uncertainties endemic in evolving digital platforms, and the credibility and legitimacy challenges of integrating this knowledge into conventional governance structures. 
%
% This work also examines and evaluates the private, transnational information sharing networks critical to mitigating and remediating global operational and security externalities and how these private intelligence networks can effectively contribute to cybercrime investigations and prosecution.

As a Postdoctoral Cybersecurity Fellow at Stanford, I was the primary author of two grants (totaling \$125,000) that funded the initial work on my ongoing Combined Capabilities Project.
%
This work evaluates how transnational groups of cybersecurity experts, international law enforcement, and intelligence communities combine their capabilities and capacities to combat global cybercrime and international security threats.
%
Participants frequently describe these relationships in terms of ``trust,'' which is important to understand, but has little explanatory power.
%
In my cumulative report for the Stanford portion of this project, I evaluated these ad hoc relationships in terms of credibility and legitimacy norms within closed, often hidden, private intelligence groups; across collaborative relationships between these actors and law enforcement; and the challenges to developing credibility and legitimacy in domestic and international systems.  
%
I am currently contributing to a report for Europol (along with colleagues at the Shadowserver Foundation) that offers a detailed retrospective on the technical, legal, and international collaboration challenges faced in the recent Avalanche takedown, involving more than 40 jurisdictions.
%
Building on this work and my Combined Capabilities Project, I am also developing a journal article on the evolution of these collaborations, from collaboration in the Conficker Working Group (2010) to the Avalanche case.

My work on reputation and influence examines monitoring and enforcement in a private transnational cybersecurity regime.
%
I evaluate how cybersecurity communities share security information to develop indicators of actors' reputation for compliance with (private, transnational) cybersecurity norms and best practices, and how these actors distribute and apply these indicators to create decentralized enforcement mechanisms.
%
An early overview of this work is available on  \href{https://www.lawfareblog.com/role-norms-internet-security-reputation-and-its-limits}{Lawfare}.
%
Information and threat intelligence sharing used to develop reputation indicators is also critical to cybersecurity incident investigations, developing forensic narratives for dissecting botnets supporting various crimeware-as-a-service platforms, and for crimeware infrastructure takedowns such as Avalanche.
% 
My early work on reputation was based on interviews amongst private actors managing these systems to characterize and understand the technical and institutional mechanisms. 
%
I am currently integrating approximately 5 years of reputation indicators (with a 4 hour sample rate), using cluster analysis to quantitatively validate trends articulated in my earlier qualitative work.
%
The resulting article, targeting the interdisciplinary \emph{Journal of Cybersecurity}, explains and evaluates this sociotechnical system.
%
This summer I am planning to work with Prof. Carsten Maple's Sociotechnical Cybersecurity group, building my \emph{Regulation \& Governance} article on IoT security standards with Prof. Irina Brass in STEaPP, to adapt these reputation models to IoT security. 


%
% On the theory side, I am currently working on an article targeting \emph{International Security} that characterizes these reputation mechanisms as a mode of persistent engagement that creates both credible signals of capabilities and serves as a form of deterrence by denial and resilience.
%

A common theme across my cybersecurity research is that the substantive elements necessary for sustainable collaboration are known and available within close-knit, yet loosely organized communities, but they continue to face questions of credibility, legitimacy, and authority at the domestic and international level.
%
In my ongoing work I frame this as a disconnect between the institutional supply and demand for cybersecurity capabilities and capacities.
%
Explaining, evaluating, and demonstrating, through both empirical and theoretical work, that these private regimes do have a credible commitment to the public interest is a precursor to systematically integrating these actors into historically closed law enforcement processes.

% The common theme across my international security and cybersecurity research is that the institutional supply and demand for cybersecurity capabilities is confounded by questions of the credibility, legitimacy, and authority of these private cybersecurity regimes in the international system.
% %
% I argue  challenges to credibility, legitimacy, along with questions of authority, confound progress to more formal engagement with international law enforcement, intelligence communities, and the broader international system.


% Historially, these ad hoc collaborations have successfully mitigated a number of global cybersecurity threats.
% %
% State actors value these epistemic communities' capabilities and capacities, but investment in the organizational and institutional capabilities necessary to systematically sustain effective collaboration between crises remains a challenge for both sides.
% %
% My combined capabilities research has characterized institutional learning in these communities from early collaborations in the late 1990s to the present through archival analysis, fieldwork, and interviews.
% %
% I argue that today, the substantive elements necessary for sustainable collaboration are known and available, but outside the institutional memory of these close-knit, yet loosely organized communities, challenges to credibility, legitimacy, and questions of authority confound progress to more formal engagement with international law enforcement and intelligence communities, and the broader international system.

  

\textbf{Policy and Practitioner Engagement} \vspace{0.2 \baselineskip} \newline %
%
Stakeholder engagement is essential to my empirical and theoretical work.
%
Over the last nine years I have interviewed over 100 actors across network
operator communities, digital platform managers, and cybersecurity and hacking communities, at over 40 network operations and cybersecurity conferences around the world.
%
I have invested significant time and effort in developing rapport with these  understudied and hidden communities.
%
I am also distinctly cognizant of balancing my research obligations and my ethical obligations to research subjects, many of whom face regular threats from transnational cybercrime organizations.
%
% Integrating and contrasting the norms and values of diverse
% sub-constituencies, I paint a non-normative portrait of the
% institutional landscape, 
% % integrating and contrasting the norms and values of diverse
% % sub-constituencies within these groups, 
% offering a view into how these communities function, the challenges they face developing the capabilities and capacity necessary to collaborate with state actors, as well as the challenges faced by state actors to understand and engage with these institutions. 
%
By demonstrating I speak technical, political, and criminal justice vernaculars, I have established a reputation as an honest broker that brings a deep understanding of the complex, sociotechnical governance and management problems endemic in establishing collaborative engagement between these transnational institutions and more conventional government and law enforcement agencies.
%
Since the beginning of my fieldwork in 2012, I have developed and sustain rare (and hard won) access to diverse formal and informal institutions critical not only to combating cybercrime, but that also provide the access and empirical evidence necessary to developing theory-based understandings of the kinds of collaboration necessary for keeping pace with continuous innovation by cybercriminals.


My recent engagements have focused on (1) understanding how these transnational regimes engage in systematic, sustainable information sharing with state actors and (2) developing global strategies for promulgating cybersecurity norms and best practices.
%
% This work has created impact on both sides, and is essential to collecting the empirical data necessary to understand the coproduction of regulation and policy for managing complex, transnational infrastructures such as the Internet.
%
% Building on this network, I would like to contribute to the Blavatnik School and the TechGov Centre's efforts to develop stakeholder relationships that will inform on-the-ground institutional dynamics that shed light on the stability and security problems facing complex digital infrastructures, and how these stakeholders can engage with the policy-making community to solve global policy problems.
% My recent engagements have given me the opportunity to apply my research.
%
As a research fellow and advisor to the Anti-Phishing Working Group (APWG), I was the program chair of the 2018 Symposium on the Policy Impediments to e-Crime Data Exchange, bringing together cybersecurity experts,
lawyers, and policy-makers to highlight the GDPR as an opportunity to resolve the tensions between operational security groups, advocacy groups, and data protection authorities wrestling with tensions between privacy and security challenges.
 % on the contested interpretations of how personally identifiable data is used in cybersecurity incident investigations, setting the stage for collaborative approaches that integrate the knowledge of industry and policy experts to solve critical cybersecurity problems. 
%
My ongoing work with the APWG is focused on transforming rich forensic data available to these cybersecurity communities into evidence-based policy advise for existing international partners such as the Council of Europe's Octopus platform for information sharing and cooperation on cybercrime and electronic evidence.
%
As a senior advisor to the Messaging, Malware, and Mobile Anti-Abuse Working Group (M$^3$AAWG), starting in 2016 I worked with the M$^3$AAWG Board to redesign their Outreach initiatives, creating and leading 
programs developing anti-abuse capabilities and capacity in Latin America and the Caribbean, Asia Pacific, and Africa appropriate for each regions' culture, values, and resource endowments, including support for engagement with regulators, law enforcement, and international organizations.
%
I am also the co-chair of M$^3$AAWG's IoT Special Interest Group (SIG), working with Internet Service Providers (ISPs) to understand and evaluate the feasibiliy of IoT reputation models.
%
I have included a reference letter detailing this engagement from M$^3$AAWG's Chairman in the attached supporting documents.

% I believe these skills, experiences, and relationships  will contribute substantively to Blavatnik and the TechGov Centre's research and education objectives.
%
Developing these global partnerships has given me substantial access to technical, law enforcement, intelligence, and international organizations actively navigating processes for more effectively collaborating to combat cybercrime.
%
It also provides unique insight into the real-world challenges of developing these collaborations, valuable as pragmatic empirical evidence for both theory- and policy-relevant research.
%
% I have developed working and research networks with diverse sets of actors,  ranging from transnational firms such as Google and Facebook, to local and regional connectivity providers and platform developers.
%
% I have had the opportunity to directly apply my research on collaboration and governance, helping to create organizations that continue to develop cybersecurity capabilities and capacities regionally, in collaboration with global partners in the larger anti-abuse community, and with the policy community.
%
%
% I am excited at the prospect of bringing these relationships to the School of Public Policy.
%
Many of these organizations have substantive demand for engagement with scholars that can help them better leverage their knowledge and capabilities in the international system and global governance.
%
This access and experience, not only understanding the technical, governance, and engagement challenges, but also understanding the diverse cultural and regional challenges, also contributes to my research-led teaching.




\textbf{Teaching} \vspace{0.2 \baselineskip} \newline %
%
Understanding the social, political, and economic challenges presented by emerging trends in Internet operations, operational cybersecurity, online platforms, and cybercrime requires exposing students to contemporary, real-world problems.
%
In addition to my security research, in my current role I was also hired to develop my department's cyber policy concentration from the ground up, creating a comprehensive curriculum and development programme for masters students in international affairs, and public service and administration.
%
This interdisciplinary, research-led programme provides accessible, deep dives into digital technologies that highlights the politics of these complex systems' design, operations, and security.
%
The four courses I teach in this program have been well received by students and faculty:
%
\begin{itemize}
  \item \emph{Introduction to Cyber Policy} offers foundations in Internet technologies, infrastructure management, privacy and surveillance, encryption, consolidation, disinformation, and cybercrime, among other topics.  Given the interests of my current students in international security and cybercrime, this course integrates contemporary applications to cybersecurity across these topics.
  \item \emph{Data Science and Visualization for Policy Analysis} focuses on applying exploratory data analysis methods, such as cluster analysis and visualization, for hypothesis generation and case selection.  Although based on R, I have specifically designed this course for students with little to no programming experience, but it can be easily scaled to a doctoral level course.  I have included the flier and syllabus for this course in the attached supporting documents.
  \item \emph{Internet Infrastructure: Platforms and Politics} focuses on the governance and politics of online platforms and infrastructures that intermediate our social, political, and economic lives (such as Facebook, Google, and mobile platforms).  This is an advanced course for students interested in a deeper dive into topics such as the politics and transnational security challenges facing specific elements of the infrastructure such as submarine cables, Internet routing, and the nuanced intersection of platform economics and security.
  \item \emph{Advanced Cyber Policy} takes a deep dive into the transnational institutional landscape, political authority amongst these institutions, and co-regulatory approaches to effectively developing cyber policy, with a special focus on the challenges of balancing security, privacy, and innovation.
\end{itemize}
%


I have also led group projects and cumulative capstones in which students engage with public and private stakeholders to apply these lessons first-hand.
%
My current capstone (description enclosed in supporting materials) builds on my Combined Capabilities Project, working with the National Cyber Forensics Training Alliance (NCFTA) and the FBI to understand the challenges of collaboration between private cybersecurity actors and international law enforcement, with a focus on business e-mail compromise, ransomware, and synthetic identity scams.  I have included the description of this capstone in the attached supporting documents.


I have considerable experience teaching, developing, and evaluating technology and policy programmes, focusing on a pedagogy that integrates understanding the technical (complex system) dynamics necessary for rigorous, evidence-based technology policy analysis and prescriptions.
%
This balance prepares students to be not only effective analysts and scholars at the policy-level, but to also serve as the increasingly important bridge between technologists and state actors on topics such as privacy and surveillance, cybersecurity, disinformation and misinformation, platform politics, and transnational infrastructure management.
%
This fundamentally interdisciplinary education in technology policy is in increasingly high demand by students, academia, and public and private sector employers---I cannot count the number of times colleagues in both the public and private sector have asked me to send them good students that understand the nuance of the technologies at play, \emph{and} domestic and international policy processes.
%
I am extremely excited at the potential opportunity to further develop both the applied and theoretical elements of this curriculum with colleagues in the Department of Security and Crime Science.
%
I also welcome the opportunity to adapt my curriculum to complement existing undergraduate, masters, and doctoral curricula.
%
In particular, I believe my interdisciplinary background in engineering systems, computer science, and criminology could substantively contribute to the Centre for Doctoral Training in Cybersecurity. 
%


%
I am quite excited at the prospect of bringing my ongoing research projects, teaching, access to expert networks, and engagement initiatives to the Department of Security and Crime Science at UCL.
%
Please do not hesitate to contact me at \href{mailto:jesse.sowell@gmail.com}{jesse.sowell@gmail.com} or +1 517 214 1900 with any questions about this application.
%
Thank you for your time and interest, I am looking forward to hearing from you.

\vspace{\baselineskip}



\closing{Sincerely,}


\encl{ industry reference letter \\
       data science course flier \\
       data science course syllabus \\
       capstone description 
     }


\end{letter}

\end{document}