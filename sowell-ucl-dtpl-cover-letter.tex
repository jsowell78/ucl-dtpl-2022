
\documentclass[11pt]{letter}

\usepackage[letterpaper, portrait, margin=1in]{geometry}

% \usepackage{lineno}
% \linenumbers

\usepackage[colorlinks = true,
            linkcolor = blue,
            urlcolor  = blue,
            citecolor = blue,
            anchorcolor = blue]{hyperref}

\usepackage{paralist}

\usepackage{enumitem}
\setlist[itemize]{noitemsep}
\setlist[enumerate]{noitemsep}
\setlist[description]{noitemsep}

\longindentation=2in

\usepackage{etoolbox}

\setlength{\parskip}{0.3\baselineskip}

\makeatletter
% If pagestyle is firstpage, remove old placement of date
\patchcmd\opening{{\raggedleft\@date\par}}{}{}{}
% If pagestyle is empty, remove vertical space after addess
\patchcmd\opening{\fromaddress \\*[1\parskip]}{\fromaddress}{}{}
% If pagestyle is empty, remove old placement of date
\patchcmd\opening{\@date \end{tabular}\par}{\end{tabular}\par}{}{}


\name{Jesse H. Sowell II}
\signature{Jesse H. Sowell II}

\def\fromaddress{\null}

\address
{
  2201 Crescent Pointe Parkway, Apt. 4302 \\
  College Station, TX  77845 \\
  m: +1 517 214 1900 \\
  e: \href{mailto:jesse.sowell@gmail.com}{jesse.sowell@gmail.com}
}



\begin{document}

\begin{letter}
{
  Department of Science, Technology, Engineering, and Public Policy \\ 
  University College London
}

\opening{Dear Faculty Search Committee,}

I am writing to express my interest in the position of Lecturer in the Department of Science, Technology, Engineering, and Public Policy (STEaPP) at University College London.  
%
I am currently an Assistant Professor of International Affairs in the Bush School of Government and Public Service at Texas A\&M University.  
%
My interdisciplinary PhD in Technology, Management, and Policy from MIT's Engineering Systems Division combines computer science, international political economy, and operations strategy.
%
My work focuses on the political economy of Internet infrastructure and security, building on extensive fieldwork to explain and evaluate how epistemic communities create and sustain the knowledge and rules necessary to keep pace with technological change, demands for diverse online services, and emerging threats.
%
Integrated research, engagement, and teaching is key to understanding contemporary and emerging challenges facing Internet infrastructure governance and security, and for preparing the next generation of sociotechnical policy analysts and researchers to solve global challenges such as cybercrime, platform governance, and improving Internet infrastructure in developing regions.  
%
% My research provides significant insights into the politics of the epistemic communities managing the Internet's infrastructure and security.
%
The insights from my research on the understudied epistemic communities are essential to systematically and effectively integrating these untapped resources into evidence-based policy making and the global governance system.
%
Over the past ten years, my engagement with these communities has created  regional and global impact, including helping to build cybersecurity communities in developing regions and addressing cybersecurity data governance challenges.
%
In the last four years my teaching has integrated this work into a comprehensive, research-led cyber policy programme, blending theory, case studies, and innovative teaching methods to %
\begin{inparaenum}[\itshape (1)\upshape]
  \item develop students' critical thinking skills 
  \item applied to project-based deep dives into substantive domains that
  \item that hones the skills necessary for policy analysts and researchers to bridge the gaps between policy and technical communities.
\end{inparaenum}
%

My current and emerging research, ongoing policy engagement, and well-developed cyber policy curriculum is exceptionally well-suited to STEaPP and the Digital Technologies Policy Lab (DTPL).
%
My ongoing and recently developed projects complement existing work in STEaPP's infrastructure and development research clusters.
%
My new projects on co-regulatory approaches to combating disinformation and data governance would be a substantive contribution to the DTPL's portfolio.
%
I explicitly designed my department's cyber policy curriculum, from scratch, explicitly for social scientists from diverse intellectual backgrounds.
%
My introduction to cyber policy and data science courses can easily be adapted for an undergraduate curriculum; based on the feedback from four years of guest lectures at STEaPP, I know advanced courses will appeal to students in the digital route, STEaPP doctoral students, and students in the Cybersecurity Doctoral Training Program.
%
The DTPL and the Policy Impact Unit would ideal homes for my ongoing engagement programmes with industry and non-profit partners on cybersecurity data governance and the ongoing challenges facing collaborations between technical communities and law enforcement.
%
I believe my intrinsically interdisciplinary portfolio is a rare and valuable complement to STEaPP's mission and programmes, and that STEaPP is where I can make the most impact, enhancing my research through collaboration with others, and continuing innovate in my teaching and engagement programmes.





%
% I believe 
%
% I thrive in and embrace multidisciplinary environments, and believe my background, research, and teaching is exceptionally aligned with this posting and the development objectives of the TechGov Centre.


%
% I believe my work identifying and closing these knowledge gaps, building bridges between the communities that manage the Internet and policymakers, is exceptionally aligned with STEAPP's mission and I am excited for the opportunity to contribute to the research and policy agenda of the Digital Technology Policy Lab.


% Recently, I have applied this work to develop partnerships among transnational cybersecurity stakeholder groups and digital platform managers, helping these actors engage with the policy community on issues such as the implementation of the GDPR and institutional development enhancing connectivity and cybersecurity in regions of South America and Africa.


\textbf{Research} \vspace{0.2 \baselineskip} \newline %
%
My research strategy has always been interdisciplinary. 
%
I started my academic life in computer science as a software engineer, focusing on programming languages and network security, but soon realized technical knowledge alone was insufficient to understand the complex sociotechnical dynamics shaping Internet development and cybersecurity challenges.
%%
My doctoral research at MIT combined international political economy, operations strategy, and computer science to conduct extensive fieldwork examining on-the-ground practices and policies in Internet infrastructure management and cybersecurity.
%
I funded the last year of my dissertation work as the primary author of a Google Faculty Research Award (\$85,000). 
%

As a Postdoctoral Cybersecurity Fellow at Stanford, I won two grants (totaling \$125,000) evaluating the informal collaboration between global  cybersecurity communities and law enforcement.
%
% This work evaluates how transnational groups of cybersecurity experts, international law enforcement, and intelligence communities combine their capabilities and capacities to combat global cybercrime and international security threats.
%
I developed and executed the research plan; managed budgets and research assistants; and ran workshops and focus groups bringing together cybersecurity professionals, law enforcement, lawyers, and policymakers in the US, Africa, and Europe.
%
The cumulative \emph{Combined Capabilities} report evaluated these collaborations in terms of credibility and legitimacy norms within these groups and in the broader global governance system.
% In my cumulative report, I evaluated these ad hoc relationships in terms of credibility and legitimacy norms within closed, often hidden, private intelligence groups; across collaborative relationships between these actors and law enforcement; and the challenges to developing credibility and legitimacy in domestic and international law enforcement and governance regimes.  
%
In collaboration with colleagues at the Shadowserver Foundation, we are continuing this work in a report for Europol.
%
This report documents and evaluates the technical, legal, and coordination challenges from the Avalanche botnet takedown, one of the largest concerted applications of Mutual Legal Assistance Treaties to date.


% Broadly, my research explains and
% evaluates the norms, values, and practices (i.e. institutions) that
% sustain the core functions of the Internet's infrastructure.
%

My research has three common themes: 
  \begin{inparaenum}[\itshape (1)\upshape]
    \item the coproduction of expert knowledge, 
    \item how it facilitates the kinds of adaptation necessary to keep pace with changes in technology and emerging security threats, and, importantly, 
    \item how to integrate expert knowledge into policy development, regulatory design, and global governance processes. 
  \end{inparaenum}
%
My %
  \href{https://link.springer.com/chapter/10.1007/978-3-030-05252-2_13}{chapter on planned adaptation} %
presents a generalized model for evaluating ad hoc and systemic planned adaptation in the regulation of complex engineering systems.
%
% In collaboration with Dr. I. Brass, \href{https://link.springer.com/chapter/10.1007/978-3-030-05252-2_13}{our article} in \emph{Regulation \& Governance} evaluates existing IoT security regulation and standards and presents a planned adaptive regulatory framework that can keep pace with innovation by cybercriminals .
%
In collaboration with Dr. I. Brass, %
  \href{https://link.springer.com/chapter/10.1007/978-3-030-05252-2_13}{our article} %
in \emph{Regulation \& Governance} presents a planned adaptive regulatory framework for IoT security regulation and standards.
%
 % I evaluate successful instances of systematically integrating epistemic communities' knowledge of complex engineering systems into regulation and policy.
%
% In collaboration with Dr. I. Brass, our article in Regulation \& Governance
% argues for a novel, planned adaptive regulatory framework that closes the feedback loops between IoT security regulations and standards development and the expert knowledge of epistemic communities necessary to keep pace with innovations by cybercriminals taking advantage of low-margin,
% insecure IoT devices.
%
My %
  \href{https://www.tandfonline.com/doi/full/10.1080/23738871.2020.1754443}{article} 
in the \emph{Journal of Cyber Policy} (featured in a  \href{https://www.chathamhouse.org/2019/12/internet-consolidation-what-lies-beneath-application-layer}{panel at Chatham House} in December 2019, travel funded by the Internet Society) comparatively evaluates consolidation in digital platforms, highlighting how governance and accountability strategies employed by communities in the Internet's infrastructure preclude the predatory practices typically associated with platform consolidation. 
%
% pushes back against critiques of consolidation, arguing for governance constructs that balance accountability and efficiency, effectively mitigating the predatory practices that characterize   
%
% My article in the Journal of Cyber Policy comparatively evaluates the governance of digital platforms, contrasting the roles of control and consensus-based knowledge coproduction for monitoring predatory practices, highlighting the need for coregulatory models that integrate accountability mechanisms with monitoring capabilities to ensure fair and equitable digital marketplaces.
%
% In an article under review with International Organization (included as a writing sample) I empirically evaluate how authority accrued to the understudied epistemic communities critical to core Internet operations, contributing to the Internet 
%
In an article currently under review with International Organization (included as a writing sample), I make empirical and theory contributions to the Internet governance and epistemic communities literature.
%
Empirically, I explain how institutions coordinating the Internet's infrastructure, in the absence of state regulation, developed the rules and accrued epistemic authority.
%
My characterization of epistemic authority is a novel theoretical contribution, and I use it to offer prescriptions on how to more effectively integrate these authorities into the broader global governance system. 
%
% contribute to the literature on epistemic communities by empirically evaluating how institutions coordinating the Internet's infrastructure, in the absence of state regulation, accrued authority, and how to more effectively integrate these authorities into the broader global governance system.
%
I believe my common research themes are exceptionally aligned with STEaPP's mission to mobilise deep expertise in complex engineering systems and policy to solve wicked global policy problems.

To coordinate across recently funded research projects, I created the Internet Infrastructure and Policy Research Group (IIPRG), where I supervise four masters-level student researchers.
%
IIPRG projects include %
%
\begin{inparaenum}[\itshape (1)\upshape]
  \item the politics and governance of submarine cables critical to Internet communication; 
  \item mix-methods modeling of the relationship between types of autocracy and Internet shutdowns, with technology transfers as an intervening variable; 
  \item studies of Internet infrastructure development in developing regions, with a special focus on Africa and Latin America; and 
  \item multilevel network analyses (combining organizational and individual ties) evaluating the globally diverse institutional complex that ensures the stability, safety, and security of the Internet, identifying critical gaps between this dense institutional network and the broader global governance system.
\end{inparaenum}  
%
The submarine cables work has produced \href{https://jpia.princeton.edu/news/leveraging-submarine-cables-political-gain-us-responses-chinese-strategy}{one student-authored publication} in the Journal of Policy and International Affairs; I am co-authoring a second article on the regional economics and security of submarine cables, under review by Contemporary Security Policy.
%
The shutdowns work has produced a co-authored, five-case article on autocracies and Internet shutdowns under review by the Journal of Peace Research; to further refine the model, the sequel (in progress) takes a mixed methods approach, using hierarchical clustering to identify trends and threshold cases in shutdown global data from 2016 to 2021. 
%
These research streams would not only enhance the DTPL's portfolio with novel and impactful research, but also creates fruitful linkages with the infrastructure and development research clusters.
%
Interdisciplinary research environments are my native habitat, and I am excited at the prospect of collaborating with colleagues in the DTPL, and across STEaPP on these kinds of projects.
% 
% In addition to these work streams, through my engagement discussed below, I am developing new research streams on data governance and disinformation.
%



% IIPL projects have a distinct focus on development dynamics in developing regions.
% %
% Their projects include: 
% %
% \begin{enumerate}
%   \item the politics and governance of submarine cables critical to modern Internet communications (a publication in the Journal of Public and International Affairs on submarine cables and the South China Sea conflict; an article under review by Contemporary Security Policy characterizing submarine cable governance challenges for regional economic and security across six cases);
%   \item studies on how autocracies are using Internet shutdowns (a five-case study developing a conceptual model of the relationships between kinds of autocratic regimes and their use of Internet shutdowns in Africa, under review by Journal of Peace Research; current work is operationalizing this model, applying cluster analysis to a global dataset from Access Now on Internet shutdowns from 2016 to 2021)
%   \item an evaluation of the role of Internet infrastructure development in developing regions, with a special focus on Africa and Latin America;
%   \item a multilevel network analysis (combining organizational and individual ties) among the globally diverse institutions that ensure stability, safety, and security of the Internet
% \end{enumerate}
% %
% While each researcher has their own projects, work in the IIPL is a distinctly collaborative endeavor, convening weekly to share progress and discuss current challenges, celebrate significant milestones, and share and exchange domain and methods expertise.
% %


%
% On the theory side, I am currently working on an article targeting \emph{International Security} that characterizes these reputation mechanisms as a mode of persistent engagement that creates both credible signals of capabilities and serves as a form of deterrence by denial and resilience.
%

% A common theme across my cybersecurity research is that the substantive elements necessary for sustainable collaboration are known and available within close-knit, yet loosely organized communities, but they continue to face questions of credibility, legitimacy, and authority at the domestic and international level.
% %
% In my ongoing work I frame this as a disconnect between the institutional supply and demand for cybersecurity capabilities and capacities.
% %
% Explaining, evaluating, and demonstrating, through both empirical and theoretical work, that these private regimes do have a credible commitment to the public interest is a precursor to systematically integrating these actors into historically closed law enforcement processes.

% The common theme across my international security and cybersecurity research is that the institutional supply and demand for cybersecurity capabilities is confounded by questions of the credibility, legitimacy, and authority of these private cybersecurity regimes in the international system.
% %
% I argue  challenges to credibility, legitimacy, along with questions of authority, confound progress to more formal engagement with international law enforcement, intelligence communities, and the broader international system.


% Historially, these ad hoc collaborations have successfully mitigated a number of global cybersecurity threats.
% %
% State actors value these epistemic communities' capabilities and capacities, but investment in the organizational and institutional capabilities necessary to systematically sustain effective collaboration between crises remains a challenge for both sides.
% %
% My combined capabilities research has characterized institutional learning in these communities from early collaborations in the late 1990s to the present through archival analysis, fieldwork, and interviews.
% %
% I argue that today, the substantive elements necessary for sustainable collaboration are known and available, but outside the institutional memory of these close-knit, yet loosely organized communities, challenges to credibility, legitimacy, and questions of authority confound progress to more formal engagement with international law enforcement and intelligence communities, and the broader international system.

  

\textbf{Engagement and Impact Through Science Diplomacy} \vspace{0.2 \baselineskip} \newline %
%
My deep, novel research findings would not be possible without continuous and trusted engagement with the epistemic communities managing the Internet's infrastructure and security.
%
In the last ten years I have interviewed over 100 actors across these communities, at over 40 network operations and cybersecurity conferences around the world.
%
% I have invested significant time and effort in developing rapport with these  understudied and hidden communities.
%
% I am also balance my research obligations and my ethical obligations to  research subjects, many of whom face regular threats from transnational cybercrime organizations.
%
% Integrating and contrasting the norms and values of diverse
% sub-constituencies, I paint a non-normative portrait of the
% institutional landscape, 
% % integrating and contrasting the norms and values of diverse
% % sub-constituencies within these groups, 
% offering a view into how these communities function, the challenges they face developing the capabilities and capacity necessary to collaborate with state actors, as well as the challenges faced by state actors to understand and engage with these institutions. 
%
Since completing my PhD, my engagement is best categorized as impact-driven science diplomacy.
%
By demonstrating I speak technical, political, and business vernaculars, I have established a reputation as a trusted honest broker that brings a deep understanding of the complex, sociotechnical governance and management problems endemic in establishing collaborative engagement between these transnational institutions, policy makers and regulators, and law enforcement.
%
I have developed rare (and hard won) access to diverse formal and informal institutions critical not only to combating cybercrime, but that also provide the access and empirical evidence necessary to developing rich, theory-based understandings of the kinds of collaboration necessary for keeping pace with continuous innovation by cybercriminals.
%


% I am proud to say that my recent work integrates fieldwork and engagement with regional and global collaborators to create real-world impact.
%
% Recent engagements have focused on (1) understanding how these transnational regimes engage in systematic, sustainable information sharing with state actors and (2) developing strategies for promulgating cybersecurity norms and best practices.
% %
% This work has created impact on both sides, and is essential to collecting the empirical data necessary to understand the coproduction of knowledge necessary for effective regulation and policy for managing complex, transnational infrastructures such as the Internet.
%
As a research fellow and advisor to the Anti-Phishing Working Group (APWG), I chaired the 2018 Symposium on the Policy Impediments to e-Crime Data Exchange, bringing together cybersecurity experts,
lawyers, and policy-makers to highlight the GDPR as an opportunity to resolve the tensions between operational security groups, advocacy groups, and data protection authorities wrestling with tensions between privacy and security challenges.
%
APWG's Secretary General Peter Cassidy recently shared that a number of participants from the 2018 Symposium indicated it was one of the most impactful meetings they have attended.
%
This year we are continuing this work, planning an annual series of Cybersecurity Data and Governance Symposia to kick off in November 2022.
 % on the contested interpretations of how personally identifiable data is used in cybersecurity incident investigations, setting the stage for collaborative approaches that integrate the knowledge of industry and policy experts to solve critical cybersecurity problems. 
%
Also with the APWG (in collaboration with Dr. L. Weissinger at Tufts' Fletcher School of Global Affairs) we evaluate the perverse incentives created by ICANN's ill-conceived GDPR compliance.
%
The research findings will contribute to a collaboration with Senator Ed Markey's (D, MA) staff to develop model legislation to ensure the accessibility of data critical to cybersecurity incident response. 

% transforming rich forensic data available to these cybersecurity communities into evidence-based policy advise for existing international partners such as the Council of Europe's Octopus platform for information sharing and cooperation on cybercrime and electronic evidence.
%
As a senior advisor to the Messaging, Malware, and Mobile Anti-Abuse Working Group (M$^3$AAWG), starting in 2016 I worked with the M$^3$AAWG Board to redesign their Outreach initiatives, creating and leading 
programs developing anti-abuse capabilities and capacity in Latin America and the Caribbean, Asia Pacific, and Africa, considering each regions' culture, values, and resource endowments, including critical support for engagement with regulators, law enforcement, and international organizations.
%
I am also the co-chair of M$^3$AAWG's IoT Special Interest Group (SIG), working with Internet Service Providers (ISPs) to understand and evaluate the feasibiliy of IoT reputation models.
%
Supporting letters from APWG and M$^3$AAWG leadership are included with this application.

% I believe these skills, experiences, and relationships  will contribute substantively to Blavatnik and the TechGov Centre's research and education objectives.
%

% Developing these global partnerships has given me substantial access to technical, law enforcement, intelligence, and international organizations actively navigating processes for more effectively collaborating to combat cybercrime.
%
Working with global partners in the cybersecurity, law enforcement, and policy communities, I apply my research on collaboration and governance to the development of impactful organizations that continue to develop cybersecurity capabilities and capacities in developed and developing regions. 
% Through these applications of my  I have created regional and global impact, helping to create , 
%
This engagement provides unique insights critical to my work.
%
Understanding the real-world challenges of developing these collaborations provides rare, valuable, and pragmatic empirical evidence for both theory- and policy-relevant research contributions.
%
% I have developed working and research networks with diverse sets of actors,  ranging from transnational firms such as Google and Facebook, to local and regional connectivity providers and platform developers.
%
%
%
% I am excited at the prospect of bringing these relationships to the School of Public Policy.
%
% Many of these organizations have substantive demand for engagement with scholars that can help them better leverage their knowledge and capabilities in the international system and global governance.
%
On-the-ground work also provides unique perspectives into the diverse cultural and regional challenges facing Internet infrastructure development and security.
%
These insights facilitate both impactful, responsible engagement and contribute significantly to my research-led teaching.


\textbf{Research-Led Teaching} \vspace{0.2 \baselineskip} \newline %
%
Understanding the social, political, and economic challenges presented by emerging trends in Internet operations, operational cybersecurity, online platforms, and cybercrime requires engaging students in contemporary, real-world problems.
%
I am a third generation teacher---a passion and dedication to teaching is in my nature.
%
My pedagogy uses innovative teaching methods such as flipped classroom, peer review, and intensive dialog structured to encourage respectful, yet rigorous policy debates.
%
In my Fall 2021 course evaluations, one student wrote:
%
\begin{quote}
  \emph{This is the first time I had Dr. Sowell and I felt he did a great job of explaining complex topics to a diverse audience. I was nervous to take a class without a STEM background but this class reaffirmed my decision and prepared me for other cyber courses I'm taking in the future. He genuinely cared about students learning the material and fostered critical thinking and discussion.}
\end{quote}


In my current role I designed, developed, and deliver, from scratch, my department's Cyber Policy Concentration (CPC),% 
  \footnote{Concentrations are similar to MPA routes in STEAPP.  Our two-year masters requires students to complete two (optionally three) concentrations to graduate.} %
offering a comprehensive curriculum and development programme for masters students coming from diverse disciplinary backgrounds.
%
This interdisciplinary, research-led programme (now in its third year) provides accessible deep dives into digital technologies and the politics of these complex systems' design, operations, and security.
%
I developed and teach four of the five courses in the CPC:
%
\begin{description}[leftmargin=0.5cm]
  \item[Introduction to Cyber Policy:] Internet technologies foundations; longstanding issues such as attribution and encryption; contemporary issues such as privacy/surveillance and disinformation

  \item[Data Science and Visualization for Policy Analysis:] exploratory data analysis (clustering, social network analysis, text mining) and visualization for mixed methods hypothesis generation

  \item[Internet Infrastructure: Platforms and Politics:] deep dive into the institutional and infrastructure economics of online platforms and infrastructures

  \item[Advanced Cyber Policy:] evaluates the diverse complex of institutions shaping Internet governance through the lens of political authority and a systems approach to global governance
\end{description}
%
I also lead capstones engaging with the National Cyber Forensics Training Alliance (NCFTA) and the FBI.

Over the last four years I contributed to STEaPP's teaching portfolio with guest lectures in Risk \& Regulation and Digital (need the full name).
%
I am familiar with STEaPP's curriculum and course structure, and would love to work with STEaPP colleagues to integrate my courses into the Digital route's educational experience and identify cross-over topics with other routes.
%
The first two above can be easily scaled to advanced undergraduate courses (syllabi included in supporting documents); the latter are appropriate for advanced MPA students and can be adapted for doctoral students.
%
I am also keen to contribute to STEaPP's cumulative group projects.
%
In addition to relationships with law enforcement, I have extensive relationships with organizations such as the Cyber Defense Alliance (CDA, based in London) and the \href{https://www.globalcyberalliance.org/}{Global Cyber Alliance} (GCA, offices in London) that would be excellent partners for MPA group projects in the Digital Route.
%
I also have substantive experience with the broader dynamics of technology and policy programmes: I have recently joined the advisory board for the Program on Emerging Technologies (PoET) hosted by MIT's Political Science Department, I have participated in and helped coordinate the \href{https://www.tudelft.nl/en/tpm/about-the-faculty/organisation-chart-facts-figures/mission}{Technology, Management, and Policy Consortium for graduate research into technology and policy}, and, through these experiences, I have learned about the diverse approaches to technology and policy education through engagement with STEaPP's sibling programmes such as Engineering and Public Policy (EPP) at Carnegie Melon University and the Department of Technology, Policy and Management at TU Delft.    

% I have considerable experience teaching, developing, and evaluating technology and policy programmes, focusing on a pedagogy that integrates understanding the technical (complex system) dynamics necessary for rigorous, evidence-based technology policy analysis and prescriptions.
% %
% This balance prepares students to be not only effective analysts and scholars at the policy-level, but to also serve as the increasingly important bridge between technologists and state actors on topics such as privacy and surveillance, cybersecurity, disinformation and misinformation, platform politics, and transnational infrastructure management.
% %
% This fundamentally interdisciplinary education in technology policy is in increasingly high demand by students, academia, and public and private sector employers---I cannot count the number of times colleagues in both the public and private sector have asked me to send them good students that understand the nuance of the technologies at play, \emph{and} domestic and international policy processes.
% %
% I am extremely excited at the potential opportunity to further develop both the applied and theoretical elements of this curriculum with colleagues in  STEAPP and the DTPL.
% %
% I also welcome the opportunity to adapt my curriculum to complement existing undergraduate, masters, and doctoral curricula.
% %
% In particular, I believe my interdisciplinary background in engineering systems, political economy, and cybersecurity could substantively contribute to the Centre for Doctoral Training in Cybersecurity. 
% %


%
I am extremely excited at the prospect of bringing my ongoing research projects, teaching, access to expert networks, and engagement initiatives to STEaPP.
%
Please do not hesitate to contact me at \href{mailto:jesse.sowell@gmail.com}{jesse.sowell@gmail.com} or +1 517 214 1900 with any questions about this application.
%
Thank you for your time and interest, I am looking forward to hearing from you.

\vspace{\baselineskip}



\closing{Sincerely,}


% \encl{ industry reference letter \\
%        data science course flier \\
%        data science course syllabus \\
%        capstone description 
%      }


\end{letter}

\end{document}