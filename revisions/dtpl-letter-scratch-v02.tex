


Intro


My research focuses on the political economy of Internet infrastructure and security, in particular the norms and best practices of private transnational cybersecurity regimes and how these actors engage with state-based security organizations and law enforcement.
%
% My intrinsically interdisciplinary background integrates political economy, computer science, and operations strategy. 
%
Empirically, my work builds on extensive engagement with global expert practitioner communities that ensure the security and stability of the Internet's infrastructure and online platforms.
%
% \textbf{Revise to focus on political economy}
%
% My political economy and international policy research the evolving relationship between technical epistemic communities and policy makers, drawing on co-regulatory approaches to integrating these actors' deep knowledge, capabilities, and capacities into domestic and international policy development, design, and implementation. 
%
I not only thrive in multidisciplinary environments---I consider them my native habitat.




My applied research is driven by real-world problems in today's digital platforms---%
  real-time management of emerging cybersecurity threats; 
  digital platform governance and its implications for markets and innovation; 
  the governance of connectivity platforms in developed and developing regions; and 
  the role of industry in mitigating disinformation campaigns.
%
These kinds of `wicked issues' cannot be resolved by policy formulation and implementation by a single stakeholder group or jurisdiction.
%
Addressing these issues requires combining technical, operational, organizational, and legal capabilities distributed among equally diverse sets of state and non-state, private actors.
%
The research and policy challenge central to my work is understanding how to systematically harness the capabilities and capacities of the diverse actors that sustain the security, stability, and integrity of complex infrastructures such as the Internet to solve global policy problems.
%
% These kinds of systemic risks are characteristic of complex digital infrastructures (and more broadly, large scale complex engineering systems).
%
% Global communities of Internet operations and cybersecurity experts have demonstrated the ability to quickly identify and mitigate transnational cybersecurity incidents, but these private actors alone lack the domestic and international authority necessary to remediate through criminal investigations, legally enforced takedown orders, and prosecution.
% %
% Collaborations do occur, but remain ad hoc and informal.  
%
% A key challenge in my security work is understanding how to bridge these gaps by more systematically integrating these private actors' necessary capabilities and capacities into state-based domestic and international processes and governance systems.
%  contributing to not only the security and stability of today's Internet, but also bringing these private actors' unique capabilities and capacities to bear on new and continuously evolving cybersecurity threats such as crimeware-as-a-service, transnational fraud campaigns, industrial and political espionage, and disinformation campaigns.

My work addresses two key parts of this challenge. 
%
First, my empirical work builds on established theories of institutional design and political economy to understand how transnational epistemic communities develop, and adapt, the norms, standards, and best practices they use to cope with uncertainties endemic in complex engineering systems such as the Internet and digital platforms.
%
Second, I evaluate the feedback loops necessary to systematically integrate this knowledge into the policy-making processes of international organizations and governance structures, while being cognizant of the politics of science and technology policy advice.
%
% Second, I evaluate the feedback loops necessary to systematically integrate this knowledge into governance, regulatory, and policy-making processes, while being cognizant of the politics of science and technology policy advice.
