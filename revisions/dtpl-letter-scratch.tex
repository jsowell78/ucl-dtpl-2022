

Intro


Across specific cybersecurity challenges, my work examines how communities of expert practitioners manage transnational cybersecurity incidents as they emerge in real-time, the coproduction of knowledge and threat intelligence that supports these responses, and the barriers to collaboration between these on-the-ground cybersecurity experts and law enforcement.


%
Understanding how to systematically bridge these gaps contributes not only to the security and stability of today's Internet, but is also critical for keeping pace with innovative cybercriminals that have made a business of exploiting new and evolving technologies. 


My fieldwork on cybersecurity and cybercrime focuses on the reactive, ad hoc relationships between private cybersecurity experts and law enforcement (both domestic and international), investigating how cybersecurity actors collaborate across firm and jurisdictional boundaries to identify emerging threats as they evolve, and the challenges facing collaboration with law enforcement.



My formal entry into the social sciences was through criminology and criminal justice, focusing on cybercrime, with an interest in informal norms and pro-social rule-breaking in the development of cybersecurity policy and best practices.



Current projects in my Internet Infrastructure and Policy Lab 
%
\begin{enumerate}
  \item the Internet Infrastructure Mapping Project, examining the role of network interconnection platforms and submarine cables in the development and security of knowledge economies through case studies, mapping, and network analysis; 
  \item the Combined Capabilities Project (elaborated below, funded at \$30,000); 
  \item work on Internet consolidation (\href{https://www.tandfonline.com/doi/full/10.1080/23738871.2020.1754443}{published} in the \emph{Journal of Cyber Policy} and presented at \href{https://www.chathamhouse.org/2019/12/internet-consolidation-what-lies-beneath-application-layer}{Chatham House} in December 2019); and 
  \item IoT security and reputation, based on applications of my planned adaptation work (\href{https://link.springer.com/chapter/10.1007/978-3-030-05252-2_13}{book chapter}) in a \href{https://onlinelibrary.wiley.com/doi/full/10.1111/rego.12343}{recent article} on adaptive regulatory governance design for IoT security standards with Dr. Irina Brass (UCL STEaPP) in \emph{Regulation \& Governance}.
\end{enumerate}%



My work examines the consensus-based coproduction of these institutions' expert knowledge, how it sustains the capabilities and capacities necessary to adapt to the uncertainties endemic in evolving digital platforms, and the credibility and legitimacy challenges of integrating this knowledge into conventional governance structures. 


%
% This work also examines and evaluates the private, transnational information sharing networks critical to mitigating and remediating global operational and security externalities and how these private intelligence networks can effectively contribute to cybercrime investigations and prosecution.


% My work characterizes the institutional capabilities and capacities of the understudied communities managing common platforms critical to Internet stability and security.


My work on reputation and influence examines monitoring and enforcement in a private transnational cybersecurity regime.
%
I evaluate how cybersecurity communities share security information to develop indicators of actors' reputation for compliance with (private, transnational) cybersecurity norms and best practices, and how these actors distribute and apply these indicators to create decentralized enforcement mechanisms.
%
An early overview of this work is available on  \href{https://www.lawfareblog.com/role-norms-internet-security-reputation-and-its-limits}{Lawfare}.
%
Information and threat intelligence sharing used to develop reputation indicators is also critical to cybersecurity incident investigations, developing forensic narratives for dissecting botnets supporting various crimeware-as-a-service platforms, and for crimeware infrastructure takedowns such as Avalanche.
% 
My early work on reputation was based on interviews amongst private actors managing these systems to characterize and understand the technical and institutional mechanisms. 
%
I am currently integrating approximately 5 years of reputation indicators (with a 4 hour sample rate), using cluster analysis to quantitatively validate trends articulated in my earlier qualitative work.
%
The resulting article, targeting the interdisciplinary \emph{Journal of Cybersecurity}, explains and evaluates this sociotechnical system.
%
This summer I am planning to work with Prof. Carsten Maple's Sociotechnical Cybersecurity group, building my \emph{Regulation \& Governance} article on IoT security standards with Prof. Irina Brass in STEaPP, to adapt these reputation models to IoT security. 
