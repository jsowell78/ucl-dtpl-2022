% \documentclass[technote,letterpaper,twocolumn]{IEEEtran} 
%\documentclass[12pt,technote,letterpaper,onecolumn,twoside,draftcls]{IEEEtran}
\documentclass[12pt,journal,compsoc,letterpaper,onecolumn,twoside]{IEEEtran}

\usepackage{comment}

\usepackage{endnotes}
\usepackage{balance}

\usepackage{hyperref}

\usepackage[T1]{fontenc}
% \usepackage{tgschola}

\usepackage{textcomp}

\usepackage{graphicx}

\usepackage{apacite}
\usepackage{paralist}

% \date{\today}

% \oldstylenums{1234}

\begin{document}

\title{Public Policy and Knowledge Assessment in
  Internet Infrastructure Resource Management} 
\author{Jesse H. Sowell}


% \author{Michael~Shell,~\IEEEmembership{Member,~IEEE,}
%         John~Doe,~\IEEEmembership{Fellow,~OSA,}
%         and~Jane~Doe,~\IEEEmembership{Life~Fellow,~IEEE}% <-this % stops a space
% \IEEEcompsocitemizethanks{\IEEEcompsocthanksitem M. Shell is with the Department
% of Electrical and Computer Engineering, Georgia Institute of Technology, Atlanta,
% GA, 30332.\protect\\
% % note need leading \protect in front of \\ to get a newline within \thanks as
% % \\ is fragile and will error, could use \hfil\break instead.
% E-mail: see http://www.michaelshell.org/contact.html
% \IEEEcompsocthanksitem J. Doe and J. Doe are with Anonymous University.}% <-this % stops an unwanted space
% \thanks{Manuscript received April 19, 2005; revised September 17, 2014.}}


% % The paper headers
% \markboth{Journal of \LaTeX\ Class Files,~Vol.~13, No.~9, September~2014}%
% {Shell \MakeLowercase{\textit{et al.}}: Bare Demo of IEEEtran.cls for Computer Society Journals}
% % The only time the second header will appear is for the odd numbered pages
% % after the title page when using the twoside option.
% % 
% % *** Note that you probably will NOT want to include the author's ***
% % *** name in the headers of peer review papers.                   ***
% % You can use \ifCLASSOPTIONpeerreview for conditional compilation here if
% % you desire.

% for Computer Society papers, we must declare the abstract and index terms
% PRIOR to the title within the \IEEEtitleabstractindextext IEEEtran
% command as these need to go into the title area created by \maketitle.
% As a general rule, do not put math, special symbols or citations
% in the abstract or keywords.
\IEEEtitleabstractindextext{%
\begin{abstract}

My research started with the inquiry ``Who
manages the Internet infrastructure and how?''  
%
% My research evolved into ongoing studies of the political economy
% of Internet operations.
%
% My work is rooted in a novel combination of technical mechanics \emph{and}
% political economy. 
%
% The answer: 
Participants in a transnational constellation of function-specific, non-state
institutions, referred to as the Number Resource System (NRS), sustain
resources supporting Internet connectivity.
%
The how: NRS participants have historically operated ``underneath the hood,'' using
consensus-based knowledge assessment processes to combine lessons from
daily operations \emph{and} ad hoc
crisis management into, thus far, stable norms and rules
that sustain the integrity of the
Internet's numbers and routing system. 
%
% The integrity of Internet connectivity 
% % critical to national economies 
% is sustained by a constellation of transnational,
% function-specific institutions, referred to here as the Number
% Resource System (NRS).
% %
% % The NRS 
% % has developed a family of credible knowledge assessment processes
% % that maintain the norms and rules supporting both ad hoc crisis
% % management and institutional adaptation apace with the Internet's infrastructure.
% %
% Thus far the NRS has functioned effectively
% ``underneath the hood,'' demonstrating ad hoc crisis management
% capabilities, rooted in adaptive, consensus-based credible knowledge
% assessment processes used to ensure norms and rules keep pace with
% innovation in the Internet's infrastructure. 
%
While successful thus far, renewed regulatory and policy interest in Internet connectivity and
security is introducing political demands historically outside the
technical and operational scope of
the NRS.
%
% Learning how to 
% \begin{inparaenum}[\itshape 1\upshape)]
%   \item navigate the global \emph{political} arena,
%   \item develop political legitimacy, and 
%   \item better engage with conventional
%     enforcement agencies
% \end{inparaenum}
% are new political capabilities the NRS must develop.
%
International relations
orthodoxy asserts that any authority not rooted in the state is
\emph{competing with} the state.
%
States covet the NRS's adaptive capabilities, 
but predatory rule threatens to undermine the
contingent authority and knowledge assessment processes that sustain
those adaptive capabilities.
%
I argue for an alternative political arrangement: operational
authorities' anticipation and 
adaptation capabilities can be \emph{complementary to} states'
capabilities.
%

% Conventional international relations asserts that any authority not
% rooted in the state is \emph{competing with} state authority---the conclusion
% of my dissertation, and ongoing studies, argue that operational authorities,
% especially their anticipation and adaption capabilities, can be quite
% \emph{complementary} to state authority.
%


% My work builds on property rights and common resource theory to
% explain how these operational institutions, highlighting these
% institutions  

Arguing for complementarity requires understanding the internal  
political mechanics of the NRS.
%
Routing standards and tacit operational
knowledge are the foundations of \emph{technical} mechanics.
% that ensures connectivity within the network of (largely) private
% networks that comprise the modern Internet.
%
% My dissertation explains the constellation of institutions that manage
% the complex of common and private resources underlying the numbering
% and routing system.
%
% To understand political mechanics, operational practices are
% operationalized in terms of 
I reframe these foundations in terms of political economy
theory---property rights, 
externalities, epistemic communities, and common
resource management---to develop an integrated vernacular
for examining and explaining the NRS's \emph{political} mechanics.
% within and across NRS institutions.
%
% The common, theory-based vernacular developed in the dissertation is
% used to explain the 
% interdependencies and tensions within the NRS. as a constellation of
% function-specific operational \emph{authorities}.
%
Empirical studies, firmly grounded in fieldwork,
use this vernacular to explain and evaluate family resemblances
within and across NRS institutions: 
%
% Bridging the gap to institutional mechanisms, management of these
% technical elements is framed in terms of property rights and common
% resource management (in the sense of Ostrom).
%
% These foundations provide the conceptual baseboards for empirical
% studies that 
% describe, explain, and evaluate a transnational complex of non-state,
% function-specific institutions that collectively ensure
% the integrity of connectivity within the Internet's infrastructure.
%
\begin{inparaenum}[\itshape 1\upshape)]
  \item uncertainties manifest as negative externalities, 
    damaging resource systems managed by the NRS;   
  \item consensus processes draw on knowledge commons and operational
    experience to develop rules and
    norms for characterizing, mitigating, and
    remediating these externalities;   
  \item NRS participants engage in \emph{mutual} 
    enforcement to deter negative externalities.
\end{inparaenum}
%
% Taken together, the NRS is a transnational constellation of
% function-specific institutions that 
% assesses uncertainties in the numbers and routing system and that has
% successfully adapted to sustain the
% integrity of the Internet's numbering and routing system.
%
Not only adaptive, these feedback processes adapt \emph{apace with}
infrastructure innovation.

Consensus-based knowledge
assessment is at the heart of these capabilities.
%
Consensus processes evaluate and adapt (common) resource
management rules to
cope with uncertainties stemming from changing resource demands and
patterns of use in the broader Internet industry.
%
Across the NRS, politically fungible voting
is eschewed as a knowledge assessment process.
%  in favor of reconciling constructive conflict
% among domain experts.
%
Under the family of consensus processes here, asserting (voting) ``No, I
do not agree,'' is insufficient. 
%
Credible contributions require a rationale that, through evaluative
policy dialogues 
that \emph{encourage} constructive
conflict, either fit or update communities'
authoritative knowledge of resource system dynamics.
%
% Contributions that fail this first test are not considered, they do
% not contribute to the ongoing consensus.
%
Shallow compliance via a marginal ``51--49 victory'' is not consensus.
%
Mutual enforcement requires credible commitments from
participants that must both adhere to \emph{and} enforce community rules.
%
As such, consensus often demands $>70$\% of credible contributors agree,
building commitment by reconciling technical critiques with dissenting minority
contributors.


Faced with this relatively foreign consensus process, regulators and policy makers familiar with fungible
voting economics and the political dynamics of issue-tying create legitimacy
issues for the NRS.
%
My ongoing public policy work addresses this problem by 
characterizing  successful, albeit nascent, coordination mechanisms
between conventional and operational authorities, generalizing these
to design more durable mechanisms that 
\begin{inparaenum}[\itshape 1\upshape)]
  \item foster familiarity with consensus processes,
  \item avoid perceived competition over authority, and 
  \item enhance complementarity across authorities' distinct 
    capabiltiies. 
\end{inparaenum}
%
My two current research programs, understanding
interconnection platform economics (Section~\ref{sec:topology}) and characterizing malware value
networks (Section~\ref{sec:mvn}), continue my industry fieldwork
to elicit tacit knowledge valuable to 
policy makers (characterizing institutional supply), then frame 
this knowledge in a vernacular familiar to 
policy makers (stimulating demand for
operators' capabilities).
%
For instance, the malware value networks (MVN) model builds on rich technical case studies
of mitigation and remediation
familiar to 
operational security communities, but reframe the dynamics as value
networks akin to organized crime or drug trafficking networks already
familiar to policy makers and law enforcement.


 


% A key capability across this institutional complex, and the source of
% adaptation rooted in access to unique operational knowledge, is the
% family of credible
% knowledge assessment processes that have been developed to cope with
% the endemic uncertainties that arise in the operation of the
% Internet's numbering and routing system.
% %
% Returning to the challenge above, knowledge assessment is 

% In contrast to conventional international relations orthodoxy that
% states any authority not rooted in the state is \emph{competing} with the
% state, I argue that the NRS's unique operational authority can be
% \emph{complementary} to state authority. 
% %
% My ongoing work focuses on how this institutional complex's deep operational knowledge and knowledge 
% assessment processes can serve as credible policy
% advise to state actors concerned with the security and stability of an
% Internet.
%






% My dissertation work is rooted in qualitative methods for developing
% case studies and extensive fieldwork.
% %
% Much of my teaching experience is in the policy realm.
% %
% My teaching statement (Section~\ref{sec:teaching}) describes my
% teaching experience, courses I would like to develop or contribute to,
% and my general teaching philosophy.
% %
% I am particularly interested in developing students'
% qualitative methods capabilities as a complement to more traditionally
% widespread quantitative methods.
% %
% I believe this will contribute to a greater understanding of how
% methods like industrial anthropology can 
% be applied to evaluating industry structures, how knowledge is
% created, evaluated, and disseminated by industry actors, and how this
% knowledge can be brought to bear in credible risk analyses and public
% policy development.


  
\end{abstract}

% % Note that keywords are not normally used for peerreview papers.
% \begin{IEEEkeywords}
% Computer Society, IEEEtran, journal, \LaTeX, paper, template.
% \end{IEEEkeywords}
}


\maketitle


% \section{Research Statement: Security Dilemmas and Knowledge
%   Assessment in Internet 
%   Infrastructure Resource Management} 

\IEEEPARstart{I}{n broad strokes,} my research can be framed as the industrial political economy of
Internet operations.
%
Who precisely are the actors underneath the hood?
%
Is ``ad hoc'' crisis management stable?
%
What are the institutions at play?
%
How do these institutions sustain credible knowledge assessment
necessary to keep 
pace with innovation in the Internet's infrastructure?
%
Are these non-state actors' incentives aligned with the public interest?
%
What are the challenges to these unique operational authorities engaging
with conventional state authorities and policy makers in national and
global political arenas?
%
The following sections offer my answers to these questions by describing   
%
\begin{inparaenum}[\itshape 1\upshape)]
  \item the distribution of operational management capabilities across
    the NRS and conventional state authorities; 
  \item the outcomes of my dissertation, summarizing theory and
    empirical contributions that contribute to explaining credible knowledge
    assessment; and  
  \item two recent research programs I have been developing to bridge
    the gap between epistemic knowledge sustained by operational
    authorities and public policy 
    makers. 
\end{inparaenum}

\section{Operational Management Capabilities, Security, and
  Knowledge Assessment} \label{sec:capabilities}

In 2008 Pakistan attempted to censure its citizens' access to YouTube.  
%
Routes intended to blackhole requests originating 
\emph{in-country} leaked,
creating an externality that blocked YouTube access \emph{globally}. 
%
In \emph{three hours} the transnational network operator community restored
the legitimate route, and, subsequently, global access to YouTube.\footnote{See \citeA{brown2008pakistan} for a step-by-step
  analysis.}  
%
To the casual observer, YouTube-Pakistan is yet
further evidence of a ``self-healing'' Internet. 
%
Among operators,
YouTube-Pakistan was just another day ensuring routing system
integrity.\footnote{The
  Pakistan-YouTube narrative is a canonical case of resource
  hijacking.  It is a well-known phenomena in the routing system; see
  \cite{toonk2014turkey, carstensen2014googles} for discussion of 
  recent events.  \citeA{cowie2008brazil} asserts that ``[i]n
  practice, most leaks [hijacks] tend  to fall into two categories,
  depending on whether they propagate: well-localized to a single
  regional customer space, or planetwide.'' 
  My dissertation characterizes leaks and hijacks as two of multiple
  classes of operational and security externalities facing the routing
  system.}
%
Although seemingly ``organic,'' such ad hoc crisis management is firmly
rooted in institutionalized norms.
%
These norms are continuously developed and
adapted by 
operators, underneath the hood, based on operational insights into
emergent uncertainties, yet apace with Internet growth. 
%
\textbf{\emph{The result: adaptive non-state institutions are \textbf{the}
  agents ensuring Internet resource integrity critical to national economies.}}


States covet operators' resource management capabilities, but
fundamentally misunderstand the function-specific constellation of
authorities sustaining resource management and 
the knowledge assessment capabilities necessary for real-time
mitigation of externalities such as YouTube-Pakistan.
%
Public facing ``Internet
governance'' institutions, such as the Internet Corporation for
Assigned Names and Numbers (ICANN) and the Internet Governance Forum
(IGF), exacerbate this misunderstanding,
bandying about 
terms like ``cyber'' and ``multistakeholderism,'' confounding
transnational social activism and cyber-libertarian ideals perpetuated
\emph{on} 
the Internet with more narrowly-scoped, non-discriminatory
resource management issues the NRS deals with \emph{in} the Internet's
infrastructure.\footnote{This assertion warrants some nuance.  The
  dissertation focused on the institutions managing the ``glue'' that
  holds the Internet together as a network of (largely) private
  networks, together---the Internet Protocol (IP) numbering (address)
  and routing 
  system, referred to here as the Number Resource System (NRS).
  Although ICANN's name implies management of numbers, it has
  historically played a titular role coordinating number rights
  delegations to the regional registries, where substantive-purposive
  authority, in the sense of \citeA{flathman1980authority}, to
  delegate numbers and develop 
  resource policy resides.  See Chapter~5 of \citeA{sowell2015finding}
  for an extensive discussion. 
  Routes describe 
  how to direct 
  traffic, not how traffic is priced and/or the discriminatory character
  of traffic management.  Issues of discriminatory access to infrastructure
  resources, such as zero-rating, are interesting topics in the larger scope
  of my research, but were not the focus of the dissertation work.} 
%
States and public policy makers see adaptation and 
resilience in the outcomes of infrastructure management decisions and
ad hoc crisis management, but
the means to harness the 
attendant authority, or even engage the NRS's operational authorities, has remained largely elusive. 
%
One-off solutions can be found, but they are fragile; ongoing work
(discussed shortly)
aims to learn from these to design more durable solutions.


A key global policy dilemma in Internet infrastructure management is
reconciling the distribution of 
management capabilities and access to commonly managed resources among
state and operational authorities.
%%
% The linchpin in this problem is  sufficiently adaptive knowledge
% assessment processes necessary to keep pace with the innovation
% continuously driving changes in the Internet's infrastructure topology.
%
Efficacious management of the NRS requires 
\begin{inparaenum}[\itshape \upshape(1\upshape)] 
  \item access to commonly managed resources and private
    infrastructure;  
  \item tacit operational knowledge held almost exclusively by industry; 
  \item resource policy making processes that adapt apace with
    innovation in the Internet's infrastructure;
  \item legitimacy in national and global political arenas; and 
  \item credible enforcement capabilities.
\end{inparaenum}
%
For the most part, the non-state resource management institutions in
which industry operators
participate hold the
first three 
capabilities; conventional policy makers and state actors hold the latter two
capabilities.
%
Stated in classical economic terms, each of these two distinct types
of political 
authorities, conventional state authorities and transnational
operational authorities, have a bundle of capabilities (an endowment) that is
necessary, but not sufficient, for sustainable management of a global,
non-discriminatory Internet numbering and routing infrastructure.


Digging into the nuance of the dilemma facing policy makers, regulators, and law enforcement 
is how the second and third
of these capabilities function, how
they are sustained, and most importantly,
how  state-based actors and operational institutions can \emph{jointly} apply
their capabilities to sustain a non-discriminatory Internet.
%
A number of questions fall out of these dilemmas:
%
If these capabilities are not merely ad hoc responses, what are the
institutions at play?
% \footnote{These are elaborated briefly in the
%   next section.}
%
Are those institutions stable?
% \footnote{Yes, thus far they are
%   stable.  The dissertation frames these institutions as a common
%   resource system.  Chapter~2 of \citeA{sowell2015finding}
%   reoperationalizes Ostrom and others work on common resource theory
%   for application to partially rival information commodities (numbers
%   and routes).  Ostrom's criteria for common resource systems
%   \cite[pp. ]{} is
%   combined with notions of authority from \cite{lake} and
%   \cite{flathman} to explain how authority has functioned thus far and
%   challenges to that authority; see Chapter~ of \citeA{sowell2015finding}.} 
%
What is the character of political authority around which operational
institutions' capabilities cohere?
% \footnote{Building on \cite{} and
%   \cite{}, authority in the NRS is contingent, substantive-purposive,
%   rooted in social capital.  The challenge, documented in the
%   literature on common resource management, is transposing social
%   capital into political capital in national and global political arenas.}
%
How does this mode of political authority promote and sustain the
credible knowledge assessment necessary for adaptive policy making?
%
How can operational institutions and conventional political actors
coordinate to develop stronger guarantees that 
the \emph{necessary} bundle of capabilities will be sustained and
continue to be applied in the public interest?\footnote{In his study
  of infrastructure writ broadly, \citeA{frischmann2012infrastructure}
  indicates that infrastructure platforms are inputs into a broad
  array of public, private, and social goods.  It is in this sense,
  whether management of the Internet's numbers and routing
  infrastructure, is maintained in the public interest.  In effect,
  the more nuanced question is whether changes in the management of
  this infrastructure diminishes the diversity of goods for which
  infrastructure platforms serve as a critical input.}



My dissertation work (Section~\ref{sec:dissertation}) addresses
the first four questions;
ongoing 
research builds on these foundations to
build bridges between operational policy entrepreneurs managing the
infrastructure and state-based policy entrepreneurs willing to
coordinate (see Section~\ref{sec:ongoing}).
%
My dissertation reveals that the scope and function of
resource management authorities are \emph{complementary to},
rather than \emph{competing with}, state authority. 
%
Ad hoc crisis management, credible operational knowledge assessment,
and adaptive rule-making are NRS authorities' strengths.
%
The social
capital undergirding these strengths does not immediately translate to
the political capital necessary to engage economies dependent on
Internet communication or to satisfy those economies' demands for
stronger guarantees of infrastructure integrity. 
%
I apply established
theory in property rights, common resource
management,\footnote{Theory and frameworks on property rights are drawn 
  from \citeA{ostrom1990governing}, \citeA{ostrom1996formation}, and
  empirical studies in \citeA{cole2012property}, among others.} and
political authority\footnote{Notions of political authority draw on
  relational authority presented by \citeA{lake2006relational,
    lake2009relational, lake2010rightful}; notions of
  formal-legalistic in contrast to substantive-purposive authority by
  \citeA{flathman1980authority}, and notions of predatory rule offered
  by \citeA{levi1989rule}.} to explain
instances of nascent complementarity identified in empirical studies.
%  yielding a
% framework for developing and offering issue-specific
% prescriptions. 
%
For instance, Operation Ghost Click\footnote{Dr. Paul Vixie, one of the
  architects of Operation Ghost Click in the operational security
  community, provides an 
  overview to the Senate Subcommittee on Crime and
  Terrorism~\cite{vixie2014testimony}.  Vixie was interviewed for
  my dissertation work and has been instrumental in facilitating
  access to
  operational security communities.}
and operators' 
``gift boxes''\footnote{Closed security groups with advanced technical
  forensic capabilities have learned to offer law enforcement ``gift
  boxes.''  These packages comprise evidence of illegal activity and
  instructions for reproducing this evidence within law enforcement's
  sanctioned chain-of-evidence procedures.  Such gift boxes allow
  closed security groups to offer assistance without becoming
  entangled as witnesses, exposing the informal social capital that
  facilitates the collection of such evidence.  While effective, this
  is an instance of a fragile process that has garnered operators some
  political capital, but relies heavily on informal social capital.}
supporting law enforcement investigations are halcyon instances of 
using endogenous social capital in operational institutions to 
develop valuable political capital in conventionally legitimate
political arenas.
%
While not taking away from these successes, it is important to note
these are relatively fragile, ephemeral solutions that, while
operating in conventional political arenas, still rely more on fragile informal
social capital than the political capital considered more legitimate
on the global stage.\footnote{Transposing social capital into
  political capital is a distinct problem for common resource
  systems.  See \citeA{birner2003using} for discussion.  For a
  discussion of social capital in general, in the context of climate
  change, see \citeA{pelling2005understanding}.  This work assumes
  Putnam's definition of social capital as reported by Pelling:
  ``features of social life---networks, norms and trust---that enable
  participants to act together more effectively to pursue shared
  objectives,'' (quoted in \cite[p. 310]{pelling2005understanding},
  attributed to \citeA[pp. 664--665]{putnam1995turning}). The next section
  (\ref{sec:dissertation}) describes
  how the NRS is framed as a common resource system. }
%
Insights from these nascent solutions highlight 
legitimacy problems between two different modes of political
authority, stressing the types of investment in engagement, comity,
and \emph{joint} capabilities 
necessary to achieve durable, lasting cooperation.

\section{Adaptation in Common Resource Institutions} \label{sec:dissertation}

% My research inquiry started with a simple question: ``Who manages the
% Internet infrastructure and how?''
% %
% This inquiry evolved into an evaluation of the Internet's numbering
% and routing
% system and the political economy of the 
% institutions that manage it.
% %
% This technical, operational, and institutional complex is referred to
% as the number 
% resource system (NRS).
%
The institutional complex that makes up the NRS has been largely
under-studied until now; it has happily sustained operations in the
shadow of more public facing institutions such as ICANN and the IGF.
%
Extensive fieldwork, interviews, and archival analysis was necessary
to surface, and then understand, the nuance of the operational epistemic communities,\footnote{Epistemic
  community is defined by \citeA[p. 3]{haas1992introduction} as ``a network
  of professionals with recognized expertise and competence in a
  particular domain and an authoritative claim to policy-relevant
  knowledge within that domain or issue-area.''  Academics and
  professionals such as physicians are canonical instances of
  epistemic communities whose knowledge base comes from formal education.  I
  refine this to describe \emph{operational} epistemic communities, those
  for whom authoritative knowledge is almost exclusively available only through
  operational experience.  This knowledge has also been described as
  tacit knowledge.  In the discussion on poltiical authority
  later in the dissertation, operational authority is further refined
  to distinguish 
  the contingent character of authority, a form of what
  \citeA{flathman1980authority} refers 
  to as substantive-purposive authority.}
resource rights management institutions, and knowledge
assessment processes that sustain connectivity among the network of
(largely) private networks that comprise the modern Internet.
%
% Authority in the NRS is contingent, rooted in a family of
% consensus-based knowledge 
% assessment processes necessary to adapt apace with Internet growth and endemic uncertainties.
%
The efficiency with which negative externalities such as 
the Pakistan-YouTube narrative are mitigated is compelling: network operators recognized a
\emph{global} negative externality, 
traced it to the origin, and remediated complicit networks in 
approximately three hours.  
%
On a regular basis, cooperation, among employees of nominally competing firms,
surfaces to repair damage to the \emph{commonly managed routing system on
which they all depend},\footnote{Ostrom highlights that participants
in common resource systems often depend on the integrity of the 
resource for their livelihood.  Similarly, network operators recognize
that their firms rely on the integrity of Internet to sustain their
value proposition.}  
then dissolves into the background noise of ``normal operations.''
%
% Lesser known cases of ``organic, ad hoc'' cooperation are common.
%

Crisis management may arise ad hoc, but the processes and mechanics
are rooted in  
distinct rights and obligations\footnote{Rights and obligations in the
  sense of \citeA{hohfeld1917fundamental}, stressing that no right
  exists without the incentives necessary to create an obligation on
  others to respect that right.}
among, and enforced by, NRS participants and institutions.   
%
In Part 1 of the dissertation, reoperationalizing technical mechanics
in terms of partially-rival information resources, the corresponding
resource rights, 
and the externalities that impact the utility of these
resources\footnote{See Chapter 2 of \citeA{sowell2015finding}}
drives home the fundamental incentive structures at play, laying the
foundation for why nominal competitors cooperate to repair the
numbers and routing system and why these same actors sustain
institutions such as network operator groups and anti-abuse
organizations (knowledge commons) documenting the types of
externalities that damage the integrity of the numbers and routing system.
%
Rigorously explaining the rationale and political mechanics of this
``ad hoc'' crisis 
management is the first contribution of my dissertation work,
providing the necessary foundations for the remainder of my
dissertation work and my ongoing research.
%
% General versus specific reciprocity
%

The second contribution of this work combines my empirical studies
with the conceptual vernacular established in Part 1 to
\emph{precisely} describe and explain who NRS participants are and NRS institutional 
structures.
%
In particular, this contribution highlights how institutional rules keep pace with high 
clockspeed application layer value networks.
%
NRS institutions are characteristically, and \emph{necessarily},
adaptive: each comprises a unique consensus process, animated by a
diverse set of nominal competitors, creating and adapting
function-specific rules
and processes that contribute to routing system integrity.
%
Part 2 of the dissertation elaborates empirical studies on the ``close-knit, yet loosely
organized''\footnote{This phrase is drawn from
  \citeA{ellickson1991order}, highlighting the character of informal
  institutions that offer solutions to property conflicts at lower
  transaction costs than offered by formal-legalistic authorities.}
communities that evolved into the modern NRS.
%
These institutions 
%
\begin{inparaenum}[\itshape 1\upshape)]
  \item share operational knowledge (network operator groups, NOGs); 
  \item delegate unique network identifiers (Regional Internet
    Registries, RIRs);
  \item create neutral markets for exchanging routes and traffic
    (Internet eXchanges, IXes);\footnote{The market paper is a deeper
      dive into the value created by actors jointly provisioning local
      platforms that create a loci for interconnection markets and the
      institutional norms that sustain these markets.} and  
  \item limit abusive appropriation of numbers and routing resources,
    here in service of the messaging industry (first anti-spam, later
    anti-abuse). 
\end{inparaenum}
%

NRS institutions are a contingent social
order, rooted in shared, authoritative images of system function and
externalities management.
%
Consensus-based knowledge assessment is the means of imbuing shared
images of system function with durable authority necessary to develop
rules and engender a mutual enforcement paradigm.\footnote{Again, this draws
  on Flathman's notion of substantive-purposive  authority~\citeyear{flathman1980authority}.  Under this
notion of authority, authority is not delegated in the sense of
conventional principal-agent models describing formal-legalistic
authority.  Rather, ``the ruled'' comply with substantive-purposive
authority as long as the ``ruler'' supplies a social order for which
the gains for compliance is less than the cost.  It is in this sense
that the order is contingent.}
%
Consensus processes evaluate
the performance of common resource management rules and, when---not if---necessary, adapt these rules to satisfy changing resource demands,
patterns of use in the broader Internet infrastructure industry, as
well as the attendant externalities that emerge with these changes.  
%
As a mechanism for coping with uncertainties, consensus and
information sharing within operational communities serves as an
explicit process for eliciting and evaluating tacit industry knowledge
across the diverse private firms that comprise the modern Internet.



Across the NRS, politically fungible voting
is eschewed as a knowledge assessment process.\footnote{There is a
  subtle distinction here.  Committees charged with coordinating
  consensus may vote to determine if the criteria for a consensus has
  been met based on the contributions of the broader body of
  participants.  It should be stressed that these coordinators are not
  voting on the substance of the issue, but rather, whether the
  criteria for consensus has been met or not independent of the
  substance of the issue.}
%  in favor of reconciling constructive conflict
% among domain experts.
%
Under the family of consensus processes identified in the NRS, asserting (voting) ``No, I
do not agree,'' is insufficient.
%
Credible contributions require a rationale that, through evaluative
policy dialogues 
that \emph{encourage} constructive
conflict, either fit or update communities'
authoritative image of resource system dynamics.
%
% Contributions that fail this first test are not considered, they do
% not contribute to the ongoing consensus.
%
%
Mutual enforcement requires credible commitments from a significant
proportion of 
participants that must both adhere to \emph{and} enforce community rules.
%
A marginal ``51--49 victory'' is not consensus; moreover, mutual
enforcement would not function under this level of
dissent.\footnote{Moreover still, in terms of authority, an indicator
  of legitimacy, especially under substantive-purposive authority, is
  the participants \emph{willing} compliance.  Here willingness
  implies the participant understands and accepts that the value of
  the social order is 
  greater than the value of deviance without that order.  This is not
  always the case in legal-formalistic models, where compliance
  ultimately rests on the legitimate use of force by the authority.}
%
As such, consensus often demands $>70$\% of credible contributors agree,
building commitment by reconciling technical critiques with dissenting minority
contributors.
%

In addition to maintaining an authoritative image of system function,
institutional norms and rules shaping daily operations are the pragmatic products of knowledge assessment processes.
%
Each distinct institution comprises function-specific
constitutional, 
collective choice, and operational rules\footnote{These classes of rules are
  drawn from \citeA[pp. 50--55]{ostrom1990governing}.} for 
\begin{inparaenum}[\itshape 1\upshape)]
  \item managing the facilities supporting unique identifier
    delegation and routing system function and 
  \item sustaining knowledge commons necessary for management and
    mitigation of negative externalities.
\end{inparaenum}
%
%
% Thus far, these
% institutions have served as stable loci of infrastructure management.
Anticipation and evaluation in the consensus process is essential to adaptive
capability, framed as a form of joint knowledge assessment by
operational epistemic communities.
%
Moreover, diverse representation, comprising experts across industry
sub-sectors, animated by constructive conflict amongst these experts,
mediated by consensus processes, makes for a \emph{durable} family of
\emph{credible} knowledge 
assessment processes that are
rare among conventional, politically 
charged regulatory 
arrangements.\footnote{See \citeA{mccray2010planned};
  \citeA{mccray2006adaptation}; \citeA{martin2006anticipation} for
  studies of knowledge 
  assessment and adaptation in conventional regulatory contexts.}
%
Collectively, these institutions sustain participants'
\emph{common interests} in jointly provisioned routing system stability.
%


NRS processes described thus far are largely endogenous to the
operator community.
%
Historically the NRS has operated quietly underneath the
hood. 
%
Scoped to \emph{common}
interests, adaptation and the resulting policy explicitly avoids impinging on public policy.
%
In contrast to conventional international regimes, the NRS limits
the \emph{scope of its authority} to supporting and enhancing routing
system function.
%
Thus far, the NRS's common interests have not run counter to the
public interest.
%
Nonetheless, a path-dependent history of harmonious
alignment\footnote{Harmonious is used in the sense defined by
  Keohane: interests are \emph{already} aligned, it is not the product of
  discord and 
  compromise that characterizes a cooperative
  arrangement~\citeyear[pp. 12, 51--57]{keohane2005ah}.  In this case,
  and to the point, harmony is serendipitous, not by design or
  pre-conceived intent.} between  
these common interests and the public interest does not carry the
same assurances made explicit by alignment resulting from coordination and
cooperation on jointly defined \emph{shared} interests.
%
Some states and state-sanctioned international
governance organizations see control of NRS facilities as critical to
preserving their own authority. 
%
Predatory claims\footnote{Here predatory refers to Levi's notion of
  predatory rule~\citeyear[see Chapter 2]{levi1989rule}.  The argument
  in the dissertation is 
  that predatory 
  rule by formal-legalistic authorities will undermine the already
  contingent substantive-purposive authority that forms the foundation
  of adaptive capabilities.} to stewardship of routing
system resources further 
complicates the alignment problem.  


To better understand this alignment problem, the global policy dilemma
outlined in broad strokes earlier, the concluding
chapters of the dissertation (Part 3) explore the question: ``Are the
incentives and resources of NRS 
institutions commensurate with the aggregate social loss due to a
partial, or worse yet, systemic, failure?''
%
Simply put, absent progress on the explicit assurances and reconciling
capabilities gaps, the
answer is no.
%
As implied earlier, would-be state principals also fall short.  
%
% State-based authorities are severely deficient in basic operational
% capabilities, the foundation of knowledge assessment necessary for
% anticipation and adaptation in the NRS.
%
Attempts to add, wholesale, the NRS stewardship function to a state's
portfolio of 
regulatory interests will expose NRS management processes to
powerful short-term interests\footnote{This is a canonical instance of
issue-bundling; Ostrom warns of issue-bundling as a threat to commons-based
management.} that will inevitably weaken, if not 
completely undermine, extant credible knowledge assessment and adaptive
capabilities.  
%
In effect, aggressive predatory rule would undermine 
precisely the characteristics (capabilities \emph{2} and \emph{3}
above) that make 
the NRS a valuable 
steward of the infrastructure.


This analysis is not a prediction of adaptive common resource
management doomed to failure.
%
Although neither the NRS, nor state
authorities, have sufficient capabilities to manage the infrastructure
on their own, a mix of their capabilities \emph{is sufficient}.   
%
The dissertation conclusion frames what explicit assurances
may look like and the barriers to developing those assurances.
%
In particular, the challenge is in establishing comity, a mutual
recognition of norms and authority among NRS institutions, state-based
authorities, and international governance organizations.  
%
Differences in mode of authority and perceptions of legitimacy
described earlier---state
in contrast to operational---further confound developing comity, though.

Developing political capital in the
global arena is a significant challenge  for the NRS. 
%
\emph{Potential} political
capital comprises 
credible knowledge assessment and adaptive capacity as the germ of
authoritative policy advice.\footnote{More precisely, credibility
  derived from constructive conflict among diverse experts from across
  industry subdomains is
  rooted in the social capital that animates operational
  institutions.  Translating social capital to political capital is
  not a simple task; see \citeA{birner2003using} for discussion.
  Explaining and 
  evaluating strategies for operator communities to develop  
  political capital broadly characterizes the ongoing research
  described in Section~\ref{sec:ongoing}.}
%
Barriers to explicit assurances identified in NRS studies draw lessons
from the deconstruction and 
reconstruction of knowledge in political
environments,\footnote{See~\citeA{jasanoff1987contested}.}  
instances of international epistemic
consensus,\footnote{See~\citeA{haas1989regimes} and
  \citeA{haas1990obtaining}.} and  
characteristics of elusive, but effective, 
adaptation that has survived in conventional regulatory
environments.\footnote{As before, see \citeA{mccray2010planned};
  \citeA{mccray2006adaptation}; and \citeA{martin2006anticipation}.}
%
Analytically, the dissertation argues the NRS and state authority need not be
competitors---the two can be quite complementary.  
%
Ongoing work builds on the pitfalls of previous efforts, in particular
short-term usurpation of the others'
authority.
%
The research programs discussed in the next sections are efforts at
leveraging operational epistemic communities unique knowledge
assessment capabilities to
build political capital by providing credible policy advice. 

\section{Bridging the Gap: Bringing Credible Operational
  Knowledge Assessment to Infrastructure Development and Security
  Policy Making}  \label{sec:ongoing}  

% \begin{itemize}
%   \item Building on the foundations established in the dissertation,
%     narrowing to two particular areas where knowledge
%     assessment, often buried in operational epistemic knowledge, can
%     provide credible policy advice.
%   \begin{itemize}
%     \item Characterizing the Internet's Infrastructure Topology
%     \item Monitoring and enforcement in the malware ecosystem
%   \end{itemize}
% \end{itemize}

My dissertation provides the foundation for ongoing studies
that draw on knowledge in operational epistemic
communities to better understand and develop strategies for
offering credible policy advice on significant issues facing industry
and society. 
%
Two long term research programs are presented here.
%
Each of these represents a multi-year research program that can easily
keep multiple doctoral students busy.

The first research program builds on studies of institutions sustaining neutral
interconnection platforms (Internet exchanges) to understand the
factors shaping the Internet 
infrastructure's evolving topology.
%
The Internet has grown from a simple network of ``dumb pipes'' driven
by economies of scale, focused on moving massive amounts of data over
long distances, to a
sophisticated 
complex of local, regional, and global marketplaces for provisioning
interconnection \emph{options} that lower barriers to access
application layer value networks.\footnote{Here option is used in the
  sense of a real option in engineering systems.  Real options
  introduce the \emph{flexibility} to expand a design or assemblage
  without making the full investment up-front.  Rather, it allows for
  a delay in the investment until better information on the return is
  available before ``exercising the option.''  The application to
  interconnection is elaborated in \citeA{sowell2013tprc}, the market
  paper.}  
%
These options lower the barriers to access and the transaction costs
of combining an increasingly diverse and 
disintegrated set 
of services---content delivery, caching, transport, and interconnection service
providers, to name a few---in support of application layer value
networks such as 
dynamic web applications, video
delivery, gaming, business-to-business financial transactions,
e-commerce, e-government, and others.
%
These marketplaces, often built atop neutral, regional interconnection
platforms, are shifting the industry structure from simplistic, carrier-based
economies of scale to platforms in which options facilitate economies of
scope that better fit regional economic demand.\footnote{It should be stressed that the platform is the loci 
  of economies of scope; disintegrated infrastructure  services are
  increasingly 
  commodified services that are combined to serve application layer
  value networks.}
%

%
The second research program builds on access to institutions engaged
in monitoring and
enforcement in the NRS and messaging system.\footnote{Messaging is the
  industry term for asynchronous communication platforms such as
  e-mail, instant messaging, social network messaging tools, and SMS
  ``text'' messaging.}  
%
This work is focused on developing models of malware
production, from identification of exploits to the command and
control operations used to transform security externalities into appropriable
economic and/or political value.
%
In addition to providing a holistic model of that illicit ecosystem,
this model will provide a  unified vernacular comprehensible to policy makers
as well as being consistent with realities observed by operators.
%
This model, the malware value network (MVN), will provide
policy makers 
with concepts \emph{based on} cases evaluated by operators, but
at a level of abstraction akin to familiar value networks such as  drug
trafficking networks and organized crime networks.
%
Such a model  highlights the
contexts in which government and operators' \emph{joint} capabilities can be 
applied to their shared interest in eliminating economically and politically
motivated malware infestations.
%
% shared versus common
%

\subsection{Value Networks Shaping the Topology of the Internet's
  Infrastructure} \label{sec:topology}

% \begin{itemize}

%   \item Typology of industry roles in a disintegrated Internet
%     infrastructure
%   \item Economics of interconnection platforms provisioning
%     interconnection options
%   \item Industry clusters as a general model of the 

% \end{itemize}

The baseline for understanding the forces driving
the Internet's infrastructure topology is a typology of application
layer value networks and the disintegrated infrastructure 
services that have evolved to meet the performance demands of those
value networks.
%
Over-simplified two-sided market models often focus on the tension in
the most generic value network, between
``content and eyeballs,'' content providers and the ``eyeballs'' that
consume content. 
%
A recent instance, the Netflix-Comcast conflict, illustrates the
problem: who should pay whom in a value
network where consumers are paying both sides---content aggregators
and access networks.\footnote{Access networks are consumer Internet
  service providers that are the last leg delivering content such as
  video streams.}
%
A major point of contention was over who should pay to reduce
congestion adversely
affecting consumer services.
%
As it turns out, the culprit was part of the platform.
%
Cogent, an intermediary transit provider, knew Netflix would eventually
shift its massive traffic elsewhere when the conflict was resolved so
it did not upgrade to accommodate this traffic, instead preferencing
the traffic of its longer-term customers. 
%

Unpacking such platforms in these two-sided models is critical to
understanding industry dynamics.
% 
Treating the platform as either a black box or a
\emph{homogeneous} network of networks glosses over important shifts from
economies of scale (often regional monopoly) carriers to a more
disintegrated market where economies of scope is both a better fit and
creates greater economic value. 
%
The increasingly
disintegrated infrastructure industry comprises diverse 
commoditized services that generally follow NRS norms, cohering into
the globally 
connected communication platform that application layer value networks rely
on, but as above, 
there are deviations.  
%
Neutral platforms in which
interconnection options reduce the cost of accessing this market of
commoditized services engender local economies \emph{of scope} among market
participants that can be combined to meet the demands of
applications with diverse performance demands.
%
As discussed in \citeA{sowell2013tprc}, the market paper enclosed with
this application, one value of IX participation is that
interconnection options lower the transaction costs of mixing and
matching commoditized services (for uniqueness or redundancy)  by
reducing measurement costs associated with exploring potential latent
demand.
%
Lowering these costs not only facilitates existing value networks, but
also lowers the barriers to exploring novel value networks.
%
Arguably, this latter also has the collateral benefit of fostering
innovation in both infrastructure services\footnote{For instance,
  early on, network transport services once IXes as competition.  Now
  there is a sub-market for transport services dedicated to facilitating
  access to multiple IXes as a way to diversify a firm's access to
  regional and global  interconnection markets.} and application layer services.

The next question for this research program is where these markets
of infrastructure services are located and what does the cost and risk of
investment in the attendant, \emph{local} platforms look like?
%
Interconnection platforms, in particular IXes, are the loci of this shift in
the Internet's infrastructure topology.
%
Following both the study of IXes in the dissertation and
\citeA{sowell2013tprc}, general platform architectures and
coarse-grain market outcomes can be specified.
%
What is missing is precise cost and risk models for investment in
these platforms.
%
Among conventional infrastructures---energy, transportation,
water---regulators and policy makers have a rich literature on cost
and risk models.\footnote{Whether they choose to make use of that
  literature is, sadly, a much different, and difficult, discussion.}
%
Corresponding models for interconnection platforms are largely
comprised of tacit
knowledge that is \emph{only} accessible via participation in the subset of
operational epistemic communities focused on developing and
participating in interconnection platforms.
%

I am currently working with a colocation facility and IX provider
in Boston to develop a first pass at cost and risk models for these
platforms.
%
Having extensive access to this facility, the first specification is
an effort in 
``getting generalizable variables right.''
%
I have discussed this effort with research subjects from my 
dissertation research to help generalize this model, extending it to
admit differences 
in the scope and scale of interconnection platforms around the
world.\footnote{My dissertation drew on interviews of actors
  participating in, designing, and administering IX platforms in the US,
  Europe, South America, and Asia, drawing on secondary analyses to
  discuss emerging platforms in Africa.}
%
The result is an empirically based model of platform development
costs that will help regulators and policy makers better understand
the costs of 
regional Internet infrastructure development.
%
One expected result that has been appealing to infrastructure
managers in the developing world is the potential for such a cost
model to express critical paths from the minimal platform necessary
to demonstrate the value of such a local interconnection market,
followed by well-known, 
step-wise improvements as the knowledge economy of a region grows,
demanding increased local capacity and infrastructure services.
%
Another benefit for developing regions is avoiding ``gold
plated'' interconnection platforms (typically affiliated with
development programs promoted by the ITU, the International
Telecommunications Union), that are over-provisioned and overly
complex for the target developing economy.


The third component of the work on infrastructure topology is to
contextualize application layer  value networks, infrastructure
services, and interconnection platforms within the capabilities of 
regional economies.
%
The objective is to evaluate the fit of a particular interconnection
platform to existing or emerging industry clusters in a particular
region.
%
As the Internet infrastructure industry has continued to lower
barriers to leveraging interconnection to enhance existing value
networks, entrepreneurs in this space are looking for models that can
help identify latent demand for local interconnection markets.
%
For instance, the Open-IX standards organization was initially
developed to bring European style associational membership IXes to
second tier US markets, diversifying the interconnection markets
beyond well-established loci in New York City, Ashburn Virginia,
Miami, Chicago, Dallas, Los Angeles, and the Bay Area.
%
Based on discussions with these actors, primarily the European IXes
discussed in my dissertation that have since deployed platforms
in the US, there is an appetite for models that can help identify
industry clusters in metropolitan statistical areas with latent demand
for a more robust interconnection market.
%
I am currently working on a sequel to \citeA{sowell2013tprc} that
integrates Porter's industry cluster framework to specify such models. 


\subsection{Monitoring and Enforcement: Understanding the Malware
  Value Network} \label{sec:mvn}


Conventional narratives on the costs of malware investigations focus on
\emph{mitigating the effects} of malware.
%
In contrast, remediation is fundamentally about identifying and
eliminating the value proposition, the illicit commodification and
directed commercialization of negative externalities.\footnote{Following
  Coase's early definition, an externality does not necessarily have
  malicious intent.  It may be a na{\"{i}}ve spillover effect.  In contrast,
  security externalities in software markets are intentionally
  identified and exploited to garner economic and/or political value.} 
%
Closed security groups within operational security communities, spanning
both network operations and anti-abuse operations, continuously
develop and adapt 
capabilities for tracking elements of the malware production chain as
they impact industry operations.\footnote{Impact is the observable.
  Subsequent investigations and information sharing in closed security groups,
  across firm boundaries, is where the measurables are developed and
  exchanged.  Returning to the broader work, the exchange of
  measurement strategies is part of the knowledge commons produced by
  the anti-abuse and security operations communities.} 
%
This work builds on articulations of these
experiences\footnote{Articulations in informal interviews,
  presentations in closed meetings, and secondary analyses published
  by security firms.} to develop a holistic view of the
malware production and distribution ecosystem, referred to
here as the \emph{malware value network (MVN)}.
%
To tap this rich operational knowledge base, this work mixes
surveys, 
qualitative case studies, and quantitative data collected by
operational security groups.\footnote{Access to this latter is a
  function of the rapport I developed with this community
  during fieldwork.  I am currently Vice-Chair of Growth and
  Development in the Messaging, Malware, and Mobile Anti-Abuse Working
  Group (M$^3$AAWG), one of the most prominent of these
  communities.  I am also participating in a closed group
  developing a survey of ISPs' malware mitigation and remediation
  strategies, building on the trust in this closed group to elicit
  more credible responses than public survey would elicit.}
%
A typology of operatioanl mechanisms and contexts is being developed
and refined to 
create an empirical model of the larger MVN.
%
% Cases is developed to create a holistic model of the 
% malware value network.
%
Such a value network comprises actors, relationships,
incentive structures, and value flows that characterize the diverse hidden
markets for externalities supporting the illicit malware production
industry.
%

This work combines perspectives from multiple communities
concerned with combating malware: 
computer scientists focusing on technical malware function and mechanics; 
industry actors contending with malware operations;  
state actors concerned with 
malware's impact on economic productivity, politically motivated
attacks, and end users (consumer welfare). 
%
% computer scientists traditionally involved in
% understanding how malware functions; industry actors that must
% mitigate and, ideally, remediate, operational malware networks; and
% law enforcement and regulatory agents concerned with the impact on
% citizens.
%
Computer scientists have traditionally contributed 
the mechanics of exploits, dissecting and explaining how malware functions
as a piece of software.
% \footnote{Foreshadow references to Stefan
%   Savage's work in the literature review and Manos' work.}
%
Operators' knowledge of malware operational dynamics, in particular
strategies for the 
production, deployment, and control of malware, is less
well-understood---modeling these operational dynamics is the novel
contribution that complements computer 
scientists' focus 
on technical mechanics.
%
This work draws on an under-studied source of
case studies and first-hand knowledge of the underground malware
industry.\footnote{It should be clear that first-hand knowledge means
  case studies from industry actors working to mitigate and remediate
  malware production chains.  A field study that infiltrates the
  MVN itself is a much more complicated proposition, not the least
  concern being the safety of the investigator.}
%
The third, and increasingly critical perspective contributing to this
work, involves the state 
and law enforcement agencies.  
%
A key contribution of the MVN, especially in the context of bridging 
gaps between operational knowledge and the design of regulatory
policy, is a common model and vernacular that \emph{enhances}
cooperation among 
industry-based anti-abuse communities, law enforcement agencies
(LEAs), and regulators.
%
Such a common vernacular will contribute to developing 
prescriptions for intervention that each set of actors will find both
comprehensible, credibly rooted in conceptual models they understand,
and that highlights how their particular capabilities contribute to \emph{joint}
interventions. 


% Drawing on the these three perspectives  to create a
% holistic model of the malware value network.
%
The MVN model offers three contributions.
%
First, it advances the state of knowledge on the end-to-end
operational dynamics of 
malware production \emph{as an illicit industry}, focusing on the value
propositions of malicious actors \emph{within the network} rather than
technical mechanics or 
ephemeral, reactive mitigation of negative externalities.
%
Second, the MVN provides operators and security
professionals with a common vernacular, rooted in empirical case
studies, for comparing and contrasting historical, ongoing, and
emerging malware production networks.
%
Such a framework provides a basis for developing systematic
remediation strategies across constituencies with complementary
capabilities, often contending with different
actors, roles, and effects in the MVN.
%
Finally, the MVN provides a coherent abstraction
familiar to state actors that understand existing illicit value
networks, such as drug trafficking and 
organized crime networks.
%
Such an abstraction hides technical mechanics that often clutter
public policy making, highlighting the criminogenic relationships, and
offering a common lens focusing operators and 
state actors on efforts to jointly develop more durable remediation
strategies.




\balance
% \theendnotes

\bibliographystyle{apacite}
\bibliography{Dissertation}

\end{document}
