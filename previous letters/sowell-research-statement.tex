\documentclass[11pt]{letter}

\usepackage{textcomp}

\usepackage{graphicx}

\usepackage{apacite}
\usepackage{paralist}

\usepackage[normalem]{ulem}

\usepackage{enumitem}
\setlist[itemize]{noitemsep}
\setlist[enumerate]{noitemsep}

\usepackage{hyperref}
\usepackage[letterpaper, total={6.5in, 9in}]{geometry}

\usepackage{etoolbox}

\makeatletter
% If pagestyle is firstpage, remove old placement of date
\patchcmd\opening{{\raggedleft\@date\par}}{}{}{}
% If pagestyle is empty, remove vertical space after addess
\patchcmd\opening{\fromaddress \\*[2\parskip]}{\fromaddress}{}{}
% If pagestyle is empty, remove old placement of date
\patchcmd\opening{\@date \end{tabular}\par}{\end{tabular}\par}{}{}


\signature{Jesse H. Sowell II}

\address{ Encina Hall C246 \\
          616 Serra Street \\ 
          Stanford, CA 94303-6165 \\ 
          \href{mailto:jsowell@stanford.edu}{jsowell@stanford.edu} \\ 
          m: +1 (517) 214-1900}

% \date{20 September 2017}

\begin{document}

\begin{letter}{ Faculty Search Committee \\
                Department of Science, Technology, Engineering, and
                Public Policy \\
                University College London                 
              }
\vspace{-1 \baselineskip}
\opening{Dear STEaPP Faculty Search Committee,}

I am writing to express my interest in the position of Lecturer in
Digital Technologies and Public Policy at the Department of Science,
Technology, Engineering, and Public Policy at UCL. 
%
I am currently a Postdoctoral Cybersecurity Fellow at Stanford
University's Center for International Security and Cooperation
(CISAC), with an upcoming appointment as Assistant Professor in the
Department of International Affairs at Texas A\&M's Bush School of
Government and Public Service. 
%
I believe my intrinsically interdisciplinary background---integrating computer science, criminology, international political economy, and on-the-ground (operational) cybersecurity---is exceptionally aligned with STEaPP’s research and teaching ethos.

My fundamentally interdisciplinary research integrates
contemporary public policy issues such as the real-time management of
cybersecurity incidents with novel conceptual frameworks in adaptive
policy-making and governance. 
%
Ultimately my research is motivated by the fact that, globally,
emerging digital technologies and processes rely on the work of non-state
actors that sustain the
security, stability, and integrity of the Internet's
infrastructure, highlighting the critical need to understand these
understudied institutions. 
%
Building on this research experience, I have taken active roles in
several transnational cybersecurity stakeholder groups, helping them
engage more actively with the policy community on issues such as the
implementation of the GDPR.


\textbf{Research} \vspace{0.2 \baselineskip} \newline 
My research strategy has always been intrinsically interdisciplinary. 
%
I started in computer science as a software engineer focusing on
network security. 
%
To integrate the wider social (such as the aggregate impact of system
vulnerabilities on end users) and economic challenges (such creating
neutral marketplaces for infrastructure services such as DDOS
mitigation), I pursued a PhD in Technology, Management, and Policy in
MIT's Engineering Systems Division. 
%
I integrated international political economy, operations strategy, and
the on-the-ground practice of cybersecurity to describe, explain, and
evaluate the norms, values, and practices (i.e. institutions) that
sustain the core functions of the Internet's infrastructure. 
%
One understudied network security community I identified is the
anti-abuse community, whose capabilities range from dissecting malware
to unravel illicit cybercrime operational networks to mitigating massive DDOS
attacks such as the recent IoT-powered attack on Dyn. 
%
% For instance, these actors apply adaptive reputation mechanisms to
% ensure a significant proportion of malicious e-mail transiting the
% Internet (90\% of all e-mail sent has some malicious payload) does not
% land in users' inboxes. 
% %
In contrast to Computer Emergency Response Team (CERT) communities
that respond to incidents \emph{after the fact}, these actors 
% are on the ground, 
adaptively respond to incidents \emph{as they happen} to repair the global digital infrastructure that sustains our socio-economic activity. 


My research at CISAC continues to bridge the gap between private
operational and policy communities. 
%
My primary project---\emph{Combined Capabilities in Cybersecurity Incident
Response}---evaluates how  hidden
cybersecurity communities and law enforcement combine their
capabilities, shifting the balance of cybersecurity from
mitigation-based whack-a-mole to collaborative remediation strategies 
that integrate private actors' deep technical capabilities
with legal enforcement actions. 
%
In the first five months of my fellowship, I raised \$125,000 to
support workshops, fieldwork, research assistants, and writing. 
%
One deliverable, a monograph with Dr Herb Lin, translates these
highly technical institutions' dynamics to language accessible
to policy makers, offering advice for how to better develop the
interface between 
operational practice and policy. 
%
The monograph also sets the stage for workshops bringing together
not only operators and law enforcement, but also public officials and
policy makers to engage in genuinely evidence-based rule making. 
%
My findings also contribute to theories of adaptive
governance in complex systems. 
%
I apply concepts from my
forthcoming chapter in \emph{Decision Making Under Deep Uncertainty}
to frame combined capabilities as a canonical 
instance of adaptation, that albeit ad hoc, has the substantive
elements necessary for planned adaptation. 
%
These research outcomes have also attracted the interest of the UK
National Cyber Security Centre (NCSC), with whom I am working to
replicate this engagement model. 
 
In the past year, I have shifted my focus from intensive fieldwork to
writing and research collaborations. 
%
Three immediate articles in draft are \emph{Reputation and Influence
  in Transnational Cybersecurity}, \emph{The Institutional Supply and
  Demand for Cybersecurity Capabilities}, and \emph{The New Digital
  Marketplaces}. 
%
In terms of collaborations, I have worked closely with the PETRAS IoT
Hub Standards, Governance and Policy stream. 
%
Dr Irina Brass and I are designing an IoT security standards model in
which reputation-based feedback loops contribute to more dynamic,
adaptive standards. 
%
% I also collaborated with Dr Leonie Tanczer on her panel at ISA,
% describing ISPs' key role in providing the data necessary for
% reputation-based adaptive security standards.  
% %
With Professor Carsten Maple at University of Warwick and the NCSC, we
are building on the success of the Combined Capabilities 
project to launch a project on the Future of Crime, with particular
focus on emerging threat vectors such as IoT. 
%
I am also working with Lloyd's Banking Group to develop a database of
cybercrime operational network topologies based on quantitative and
structural data from the anti-abuse community.
%
% I would be delighted to bring these interdisciplinary projects to STEaPP and I believe they will enhance the research portfolio of the Digital Policy Lab.

\textbf{Policy Engagement} \vspace{0.2 \baselineskip} \newline  
Stakeholder engagement is absolutely necessary to understand
institutional dynamics and credibly represent the real problems these
communities face. 
%
Over the last six years I interviewed over 100 actors from network
operator communities, cybersecurity and hacking communities, and
operational institutions' leadership at over 40 network operator and
cybersecurity conferences around the world. 
%
By integrating and contrasting the norms and values of diverse
sub-constituencies, I paint a non-normative portrait of the
institutional landscape  
% integrating and contrasting the norms and values of diverse
% sub-constituencies within these groups, 
that offers a credible view to both these communities and policy stakeholders. 
%
By demonstrating I speak both technical and policy vernacular, I have
established a reputation as an honest broker that brings a deep
understanding of the sociotechnical problems intrinsic to both
technical and policy issues.


I have recently taken a very active role in these communities. The
Anti-Phishing Working Group (APWG) invited me to serve as the program
coordinator for Symposia on the Policy Impediments to e-Crime Data
Exchange, bringing together operational cybersecurity experts,
lawyers, and policy makers to better understand the implications of
the GDPR for critical security operations. 
%
In a forthcoming article for Lawfare I argue that the 
GDPR is not a threat, but an opportunity to resolve the tension
between operational security groups, privacy organizations, and data
protection authorities on the contested interpretation of how personal
data is used in cybersecurity incident investigations. 
%
At the Messaging, Malware, Mobile Anti-Abuse Working Group (M$^3$AAWG)
I am the Senior Advisor Directing Outreach.
%  and the Co-Chair of the IoT
% Special Interest Group. 
%
In collaboration with M$^3$AAWG leadership, I created and lead ongoing
programs to 
develop anti-abuse capabilities in Latin America and the Caribbean, Asia Pacific, and Africa. 
%
Leveraging my knowledge of the diverse governance challenges in these
developing economies, these programs are coproducing cybersecurity
norms appropriate for each regions' culture, values, and resource
endowments, including support for government engagement.


\textbf{Teaching} \vspace{0.2 \baselineskip} \newline 
I strongly believe that the only way to truly understand complex
engineering systems, such as digital infrastructures, is to expose
students to real problems through an interdisciplinary lens that
highlights the interplay amongst social, economic, political, and
technical factors. 
%
I am extremely excited at the prospect of contributing to the Masters
in Digital Technologies and Public Policy at STEaPP. 
%
I would love to teach courses on topics such as the political economy
of cybersecurity and the policy challenges for digital technology
infrastructures in the developing world. 
%
In these courses, I would like to bring in guest experts, such as threat intelligence and infrastructure managers, that have firsthand experience with these issues and that would love to work with MPA students on practical projects. 


I also believe that I can substantively contribute to core MPA modules,
mentoring MPA students, and the Doctoral Training Program. 
%
My work on planned adaptation, credible knowledge
assessment, and risk can contribute to \emph{Evidence for
  Decision-Making} in
the \emph{Introduction to STEaPP} and \emph{Risk Assessment and Governance}
courses. 
%
I am also interested in teaching and mentoring students on the
comparative politics of national cybersecurity policies, the tension
between privacy and security, industrial political economy,
infrastructure economics, and mixed research methods. 
%
For the Bush School, I am developing a mixed methods course on cybersecurity that uses datasets from multiple threat intelligence firms that are enthusiastic to see students develop a rigorous methodological toolkit grounded in solving real world problems. 

I also have strong teaching experience developing students with pragmatic
policy analysis skills. 
%
At MIT, I was the lead teaching assistant (TA)
for two core courses in the Technology and Policy Program (TPP):
\emph{Introduction to Technology and Policy} and Professor Ken Oye's \emph{Science,
Technology, and Public Policy (STPP)}. 
%
The introduction teaches
students how to concisely and effectively convey complex
sociotechnical trade-offs to policy makers, cognizant of how technical
data and indicators are often distorted by the political process. 
%
STPP applies political economy to real world case studies on the policy
challenges and uncertainties endemic in science, technology, and
engineering, highlighting the role of market and institutional
failures, credible knowledge assessment dilemmas, and the trade-offs
between economic efficiency, ethics, and values.

% Concluding paragraph on your philosophy, suitability and contribution to STEaPP: Full package Jesse.

I believe my approach to research, engagement, and teaching is very
well aligned with 
STEaPP's mission to bridge the gaps between technology and engineering
experts, and policy makers. 
%
My work is intrinsically interdisciplinary, grounded
in the  pragmatic problems faced when managing complex engineering systems.
%
My work is not just applied: I 
re-operationalize established theories in political economy,
organizational theory, and security studies to explain the role and
capabilities of institutions managing complex engineering systems
while also navigating the domestic and international policy space.
%
I am enthusiastic about the prospect of bringing my ongoing
research projects, access to expert networks, and engagement
initiatives to STEaPP and the Digital Policy Lab, enhancing and complementing its existing
research and teaching portfolio.
%


\closing{Sincerely,}



\end{letter}
\end{document}