\documentclass{letter}[12pt]

\usepackage{textcomp}

\usepackage{graphicx}

\usepackage{apacite}
\usepackage{paralist}

\usepackage[normalem]{ulem}

\usepackage{enumitem}
\setlist[itemize]{noitemsep}
\setlist[enumerate]{noitemsep}

\usepackage{hyperref}
\usepackage[letterpaper, total={6.5in, 9in}]{geometry}


\signature{Jesse H. Sowell II}

\address{ Encina Hall C246 \\
          616 Serra Street \\ 
          Stanford, CA 94303-6165 \\ 
          \href{mailto:jsowell@stanford.edu}{jsowell@stanford.edu} \\ 
          m: +1 (517) 214-1900}

\date{20 September 2017}

\begin{document}

\begin{letter}{ Professor Gregory Gause \\
                c/o Ms. Janeen Wood \\
                The Bush School of Government \& Public Service \\
                Texas A\&M University \\
                4220 TAMU
                College Station, TX 77843-4220
              }


\opening{Dear Prof. Gause,}

I am writing to apply for the position of Assistant Professor in The
Department of International Affairs in the Bush School of Government
and Public Service in the area of cyber policy, strategy, and
security.  
%
I completed my dissertation, entitled \emph{Finding Order in a Contentious
Internet}, in January of 2015 in the Technology, Management, and Policy
track of the Engineering Systems Division at the Massachusetts
Institute of Technology.
%
I am currently starting my second year appointment as a Postdoctoral Cybersecurity Fellow at Stanford's Center for International Security and Cooperation (CISAC).

My research lies at the intersection of Internet operations,
international security, and policy development.  
%
My dissertation
examined the thus far under-studied, non-state institutions that
ensure the network of (mostly) private networks that comprise
the Internet remain stably and securely ``glued
together.''
%
My dissertation uses common resource economics to explain why these
actors, many of whom are nominal competitors, collaborate to mitigate
security externalities; how consensus-based decision making processes
serve as credible knowledge assessments that yield adaptive resource
policies apace with innovation in the Internet's infrastructure; and
the scope and limits of these institutions' management and enforcement
capabilities, in particular a reputation-based regime rooted in
deterrence by denial.
%
To date my work is based on six years of fieldwork, interviewing over 100 network operators and community leadership at over 40 conferences around the world.

Although these transnational institutions have sustained the integrity
and security of the Internet thus far, their capabilities have limits.
For instance, reputation-based deterrence by denial is sufficient to
remediate some security vulnerabilities, but for the most recalcitrant
malicious actors---organized crime, industrial espionage, and
politically motivated hacking---operators are increasingly reaching out to law enforcement to prosecute, with the hope of more permanently ending the well-known cat-and-mouse game of displacement.  I argue that, going forward, while the public sector and these private institutions have unique subsets of the capabilities necessary to remediate cybersecurity incidents, neither the private sector---in my work the transnational operator communities that have thus far mitigated Internet security externalities---nor the state, has sufficient capabilities to sustain a stable, secure, globally connected Internet.  To succeed, these two very different regime complexes must collaborate, they must reconcile different decision making processes and authoritative structures to combine their capabilities \emph{without undermining} one another.  

Currently I am examining and documenting early instances of combining
capabilities such as Operation Ghostclick and, more recently,
Avalanche in order to test and further explore the propositions
described above.  During my first year at CISAC, I was the primary
author of two proposals to support this work, bringing in \$125,000 of
funding for interviews, fieldwork, and, soon, Track-II style workshops
where state actors and operators will share their experiences, both
successes and failures, with combining capabilities.  I am engaging
directly with parties involved in operational incident response to
better document and understand 
\begin{inparaenum}[(1)]
  \item how these communities share information necessary to monitor
    and identify cybersecurity incidents, 
  \item how to balance demand for automated security information
    exchange with qualitative human intelligence necessary to identify
    novel attacks and strategies,
  \item how private intelligence can improve cyber incident response
    by developing more systematic combined capabilities with law
    enforcement and national security actors, and 
  \item how to use lessons from both successes and failures to better
    design cybersecurity policy in light of these dynamics.   
\end{inparaenum}

I have two primary publication streams in the works for the next
couple of years.  The first publication stream is part of the combined
capabilities project.  This stream will establish an academic
literature on the understudied institutions that sustain the security
and integrity of the Internet, focusing on how political mechanics
differ from conventional political decision making processes,
especially the role of credible knowledge assessment.  This
publication stream will also produce a short (70-100 page) book (with
Herb Lin) summarizing these findings and strategies for combining
capabilities into pragmatic prescriptions targeting policy makers.
The second is a series of papers on reputation: the working paper
establishing the economics of reputation and deterrence by denial, a
sequel adapting that model to apply reputation as an incentive to
improve the security of Internet of Things devices, and a third
elaborating the incentives and economics of Internet Service Providers
(ISPs) critical to the feasibility of my IoT reputation model.  I am also planning an academic manuscript based on my dissertation and ongoing work entitled \emph{Order, Political Authority, and Policy in the Internet's Infrastructure}.

I have teaching experience at both the undergraduate and graduate
level.  At MIT, I was the teaching assistant for two core courses in
the Technology and Policy Program: \emph{Introduction to Technology
  and Policy} and \emph{Science, Technology, and Public Policy}.  The
former introduces students with an engineering background to the
nuance of making technology policy, with a critical eye for how
technical data and indicators are politicized.  The latter is an
application of political economy concepts to topics such as
pharmaceutical research, synthetic biology, Internet protocols and
security, national intelligence, and food safety.  I am especially
interested in building on this experience and my work on combined
capabilities to develop a cybersecurity and policy program.  I am also
interested in teaching methods courses, both on quantitative and
qualitative methods.

In terms of departmental service, I currently run the Cyber Reading
Group at CISAC and am working with our parent institute to develop
social activities for current and returning fellows.  I am also
involved in industry groups related to my research, serving as the
Senior Advisor Directing 
Outreach for the Messaging, Malware, and Mobile Anti-Abuse Working
Group (M$^3$AAWG) and as a program committee member at the North
America Network Operators Group (NANOG). 

I have enclosed my CV and writing samples illustrating a policy
recommendation, my work on internet infrastructure economics, and a
dissertation chapter on anti-abuse (cybersecurity).  If you need any additional information, please do not hesitate to
contact me at \href{mailto:jsowell@stanford.edu}{jsowell@stanford.edu}
or +1 (517) 214-1900.

\closing{Sincerely,}


\vspace{0.5in}

\encl{ \texttt{sowell-cv-main.pdf} \\
       \texttt{sowell-writing-sample-policy-recommendation.pdf} \\
       \texttt{sowell-writing-sample-internet-economics.pdf}
       \\  
       \texttt{sowell-writing-sample-anti-abuse-dissertation-chapter.pdf}
     }

\end{letter}
\end{document}